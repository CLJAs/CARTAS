\chapter{Introducción}

\noindent
La propuesta consiste básicamente en realizar un experimento, barato, rápido y sencillo, de una duración aproximada de 4 horas y media, capaz de demostrar que el cardinal de $\mathbb{P(N)}$ NO es mayor que el cardinal de $\mathbb{N}$. No sólo con lo que todo ello implica, sino como una puerta de salida del paraíso que creó Cantor, que lleva a un mundo de conceptos nuevos y resultados increíbles, y muy necesarios. Algunos solo necesitan ser recopilados de forma adecuada, y otros requieren un equipo de investigación.

\noindent
¿Por qué un experimento y no un paper? Por muy diversas pero poderosas razones. La idea básica, y muy resumida, es que es difícil creerlo hasta que no se ve. En diversos aspectos. Se puede hablar durante dias, semanas o meses, cuando una exposición de 4 horas y media puede cerrar todos los debates. La idea del experimento es sacar al autor original de la ecuación, y que sean otros matemáticos quienes se enfrenten al fenómeno matemático, teniendo que lidiar con sus propias palabras y sus consecuencias, que encima, ellos mismos van a elegir.

\noindent
El autor original carece de prestigio, pero no de razón ni resultados. Eso lleva a muy variopintas reacciones... que hay que ver para creer. La idea base es reunir a un grupo de gente dispuesta a retar su mente y sus conocimientos preestablecidos, y observar sus reacciones al lidiar con un puzzle imposible de resolver, sin negar por el camino el teorema. Gente dispuesta a hacer público el material grabado en vídeo, y usar la distribución de ese vídeo para conseguir convenios con varias universidades, para obtener los recursos y personal adecuados para transformar dicho experimento, en un paper formal aceptado por toda la comunidad internacional, en el cual ya se recopilan TODOS los resultados y no solo un subconjunto de ellos (que es lo que sería el experimento).

\noindent
A los participantes se les ofrece a cambio, ser partícipes de un momento histórico, que será referenciado durante siglos. Y si se consiguen los recursos, ser considerados para una oferta de un puesto en el equipo de trabajo futuro. Ese equipo creará un paper que será considerado un básico en la formación de cualquier matemático en las futuras generaciones. No solo por sustituir gran parte de la actual teoría de conjuntos con muchos conceptos nuevos; no solo por cambiar paradigmas establecidos; sino también por la necesidad de encontrar nuevas demostraciones a viejos teoremas básicos y crear axiomas nuevos.

\noindent
Cualquier institución que se relacione con la creación y distribución pública de dicho experimento, tendrá una de las mejores campañas de marketing que se pueden soñar para una institución de ingeniería, ciencia o investigación. A cambio del ridículo precio de montar el experimento y un riesgo muy reducido. Ni siquiera hace falta darle mucha pompa, sino simplemente venderlo como un desafío informal, las reacciones de los matemáticos presentes hablarán por sí solas.

\newpage
\noindent
¿Qué preguntas pretende responder está presentación?:\\\\
¿Qué garantías se ofrecen para decidir si apoyar la realización del experimento?\\
¿Por qué no escribirlo en un simple paper y publicarlo?\\
¿En qué consistiría el experimento? Así se puede valorar su coste\\
Condiciones a tener en cuenta si el proyecto resulta interesante.
