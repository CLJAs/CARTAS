\chapter{Experimento vs paper}

\noindent
¿Por qué no está publicado como un paper? Pues porque no soy matemático. NO es que no esté publicado. Lo está en diferentes sitios, en diferentes formatos, diferentes partes y versiones.

\noindent
El experimento se necesita, para demostrar que el trabajo es digno de un esfuerzo, cuyo objetivo sea reescribirlo de una forma lo más estándar posible, de acuerdo a los protocolos matemáticos de la comunidad internacional.

\noindent
Mi estilo personal provoca unas fases de reacciones cíclicas... una serpiente que se muerde la cola:\\\\
I) Se leen cualquier cosa escrita de muy mala leche y deprisa (quién se lo lee), y luego afirman que TODO es una absoluta basura. A menudo usan palabras peores :D.\\\\
II) No sé cómo, pero alguien a veces se queda... y empiezo a preguntar frase por frase... y de repente, todo es mucho más claro, y ya no es tan basura. Incluso cambian de parecer, y empiezan a encontrar cosas que les parecen interesantes (varias, según la persona).\\\\
III) Encuentran su fallo aleatorio particular, se quedan contentos, y cortan comunicación. Pero cuando enseño ese mismo punto a otros matemáticos, lo consideran obvio y trivial, y me miran raro si les pido que lo confirmen dos veces... (se nota por el tono de escritura :D)\\\\
IV) Consigo que se queden hasta la conclusión final, dando okey de manera informal a cada punto. Pero una vez ven la conclusión final:\\
4a) O escogen una de las dos opciones del experimento (ya lo veremos). Lo cual sin saberlo, indica que tienen absoluta certeza que el teorema es falso, pero no lo saben.\\
4b) No pueden ver el fallo, lo admiten, pero suponen que debe existir. Y me aconsejan publicarlo para que otra persona me indique el fallo.\\\\

\noindent
Pero claro, si pudiese publicarlo en el formato y protocolos correctos, no necesitaría ni el experimento, ni la ayuda de nadie. Sin embargo, el experimento se usa para que las dos opciones de la cuarta fase, no hablen de ``mis palabras'', sino de las suyas propias. Y que contrasten lo que dicen, no con un desconocido sin estudios, sino con colegas, que dicen lo contrario que ellos, y ambos me dan la razón sin poder evitarlo.

\noindent
Y se cierra la serpiente que se muerde la cola: lo re-escribo, y mi estilo provoca otra vez la fase I, II, III...

\noindent
Si mi lenguaje y herramientas, poco ortodoxas, fuesen un problema REAL de rigor, Pepe Mendez me hubiese parado y hubiese indicado que ese era el fallo. El truco para entendernos es simplificar TANTO cada punto, que un matemático pueda decidir si lo considera correcto o no. Pero al no usar lenguaje ortodoxo, siempre les queda un mal sabor de boca al sacarles de su zona de confort.

