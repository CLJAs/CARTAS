\documentclass[12pt, notitlepage]{report}
\usepackage[left=1in, right=1in, top=1in, bottom=1in]{geometry}

%Para usar colores en las tablas:
\usepackage{color}
%\usepackage{graphicx} DUPLICADO
\usepackage{epsfig}
\usepackage{multirow}
\usepackage{colortbl}
\usepackage[table]{xcolor}
%Fin de paquetes para usar colores en las tablas


\usepackage{titling}
%\usepackage{lipsum}
\usepackage{mathtools}
\usepackage{amsmath}
\usepackage{amsfonts}
\usepackage{amssymb}
\usepackage{pdfpages}
\usepackage[spanish]{babel}
\usepackage[utf8]{inputenc}
\setlength{\parskip}{2mm}
\usepackage{graphicx}
\usepackage{hyperref}
\usepackage{comment}
\graphicspath{ {./Imagenes/} } 

%Definiendo colores:
\definecolor{lightgray}{gray}{0.9}
\definecolor{myblue}{RGB}{180,241,231}
\definecolor{myred}{RGB}{241,121,108}
\definecolor{myyellow}{RGB}{245,239,122}
%Fin de definicion de colores:


\pretitle{\begin{center}\Huge\bfseries}
	\posttitle{\par\end{center}\vskip 0.5em}
\preauthor{\begin{center}\Large\ttfamily}
	\postauthor{\end{center}}
\predate{\par\large\centering}
\postdate{\par}

\title{Propuesta para refutar el Teorema de Cantor}
\author{Juan Carlos Caso Alonso}
\date{\today}
%\date{19 January/Enero 2023}

%\renewcommand{\chaptername}{C.-}
%\usepackage{fancyhdr}

\addto\captionsspanish{\renewcommand{\chaptername}{Capítulo}}

\begin{document}
	\maketitle
	\thispagestyle{empty}
	
	%\input{./abstract.tex}
	
	\addtocontents{toc}{\hspace{-7.5mm} \textbf{Capítulos}}
	\addtocontents{toc}{\hfill \textbf{P\'agina} \par}
	\addtocontents{toc}{\vspace{-2mm} \hspace{-7.5mm} \hrule \par}

	\tableofcontents
	
	%\part{Español}
	\chapter{Introducción}

\noindent
La propuesta consiste básicamente en realizar un experimento, barato, rápido y sencillo, de una duración aproximada de 4 horas y media, capaz de demostrar que el cardinal de $\mathbb{P(N)}$ NO es mayor que el cardinal de $\mathbb{N}$. No sólo con lo que todo ello implica, sino como una puerta de salida del paraíso que creó Cantor, que lleva a un mundo de conceptos nuevos y resultados increíbles, y muy necesarios. Algunos solo necesitan ser recopilados de forma adecuada, y otros requieren un equipo de investigación.

\noindent
¿Por qué un experimento y no un paper? Por muy diversas pero poderosas razones. La idea básica, y muy resumida, es que es difícil creerlo hasta que no se ve. En diversos aspectos. Se puede hablar durante dias, semanas o meses, cuando una exposición de 4 horas y media puede cerrar todos los debates. La idea del experimento es sacar al autor original de la ecuación, y que sean otros matemáticos quienes se enfrenten al fenómeno matemático, teniendo que lidiar con sus propias palabras y sus consecuencias, que encima, ellos mismos van a elegir.

\noindent
El autor original carece de prestigio, pero no de razón ni resultados. Eso lleva a muy variopintas reacciones... que hay que ver para creer. La idea base es reunir a un grupo de gente dispuesta a retar su mente y sus conocimientos preestablecidos, y observar sus reacciones al lidiar con un puzzle imposible de resolver, sin negar por el camino el teorema. Gente dispuesta a hacer público el material grabado en vídeo, y usar la distribución de ese vídeo para conseguir convenios con varias universidades, para obtener los recursos y personal adecuados para transformar dicho experimento, en un paper formal aceptado por toda la comunidad internacional, en el cual ya se recopilan TODOS los resultados y no solo un subconjunto de ellos (que es lo que sería el experimento).

\noindent
A los participantes se les ofrece a cambio, ser partícipes de un momento histórico, que será referenciado durante siglos. Y si se consiguen los recursos, ser considerados para una oferta de un puesto en el equipo de trabajo futuro. Ese equipo creará un paper que será considerado un básico en la formación de cualquier matemático en las futuras generaciones. No solo por sustituir gran parte de la actual teoría de conjuntos con muchos conceptos nuevos; no solo por cambiar paradigmas establecidos; sino también por la necesidad de encontrar nuevas demostraciones a viejos teoremas básicos y crear axiomas nuevos.

\noindent
Cualquier institución que se relacione con la creación y distribución pública de dicho experimento, tendrá una de las mejores campañas de marketing que se pueden soñar para una institución de ingeniería, ciencia o investigación. A cambio del ridículo precio de montar el experimento y un riesgo muy reducido. Ni siquiera hace falta darle mucha pompa, sino simplemente venderlo como un desafío informal, las reacciones de los matemáticos presentes hablarán por sí solas.

\newpage
\noindent
¿Qué preguntas pretende responder está presentación?:\\\\
¿Qué garantías se ofrecen para decidir si apoyar la realización del experimento?\\
¿Por qué no escribirlo en un simple paper y publicarlo?\\
¿En qué consistiría el experimento? Así se puede valorar su coste\\
Condiciones a tener en cuenta si el proyecto resulta interesante.

	\chapter{Garantías}

\noindent
La primera garantía es que el experimento YA ha tenido éxito, pero en foros matemáticos y reuniones privadas. La idea es juntar a personas que muy probablemente van a reaccionar de la misma manera y hacerles escucharse unos a otros.

\noindent
En una conversación privada se pueden ofrecer decenas de anécdotas de todo tipo e índole con gente de diversos niveles. Desde catedráticos a estudiantes, a gente que supuestamente ha hecho su tésis con Terence Tao, físicos, ingenieros... No sirven de mucho, pq nadie cree que sean realidad. Es UNA de las razones por las cuales se necesita dejar testimonio público.

\noindent
Se pueden ofrecer resultados, y dejar que las matemáticas hablen, pero claro, eso necesita tiempo. Hay resultados más sencillos, otros más complejos, algunos curiosos, y otros, como descubrí hace unos meses, decisivos y rotundos. La SEGUNDA razón es evitar que la gente salga huyendo ante la imposibilidad de negar un resultado. Tantas reacciones diferentes que contar... una muy común es decir: ``No soy capaz de encontrar el fallo en la exposición, pero alguien más experto en el campo SEGURO que lo encuentra''. Normalmente no encontrar fallos en una exposición lleva a una recomendación de un estudio más profundo o una publicación. Nada de eso sucede.

\noindent
Como resultados, se puede ofrecer uno muy rápido, que en apenas 15 minutos se puede solventar, que es respuesta a supuestos desafíos imposibles, que una vez resueltos, acaban en excusas para no admitir que minutos antes era algo que se creía imposible. En 15 minutos puedo empezar a ofrecer ``grietas'' en las demostraciones del teorema. Son demostraciones que están públicas y algunos divulgadores las consideran correctas, sin corrección hasta que se conoce el resultado propuesto, que hasta la fecha nadie ha podido negar.

\noindent
Casi todas las demostraciones tienen el mismo defecto: son técnicas reproducibles entre conjuntos que tienen el mismo cardinal. Lo cual las hace no confiables. Y eso se manifiesta de diversas formas, con fenómenos numéricos que ya han sorprendido a diversos matemáticos. La dificultad radica en que la capacidad para comprobar que ambos conjuntos tienen el mismo cardinal, varía. Por eso se ofrece una reproducción muy sencilla y rápida, de un caso muy simple, para demostrar que SUCEDE, y crear la duda sobre si sucede en más casos, ganando así el derecho a más tiempo y exposiciones más largas.

\noindent
La demostración definitiva, es ``observar'', como el cardinal de $\mathbb{P(N)}$ y $\mathbb{N}$ son indistinguibles, presentados en el formato adecuado. Y luego estudiar TODAS las demostraciones clásicas que afirman lo contrario. Y si hay curiosidad, aparte del experimento, en otro tiempo, hablar de resultados paralelos que reflejan una y otra vez lo mismo. Y la demostración consistirá en dos grupos de matemáticos diciendo que un conjunto es más grande que el otro y viceversa, con absoluta seguridad, sin ningún tipo de duda, hasta que oigan al otro grupo decir EXACTAMENTE lo contrario :D.

\noindent
La incredulidad, y la falta de fé en la propia razón y conocimientos, son reacciones muy comunes. Aquí dejo un texto que me envió por email un catedrático al que le realicé una exposición parcial de mi trabajo:\\\\
\\
``\\
Hace ya un par de años un compañero de la Universidad me pidió que atendiera a un amigo suyo, 
que tenía unos resultados matemáticos sobre los que quería saber mi opinión. Así conocí a un 
joven, Juan Carlos Caso Alonso, que sostenía que el teorema de Cantor es falso y que el conjunto 
potencia de cualquier conjunto infinito tiene el mismo cardinal que dicho conjunto. Le advertí que 
para mí, y para casi todos los matemáticos del Mundo, el resultado de Cantor es un teorema verificado.
Juan Carlos no tiene formación matemática, pero posee unas grandes inquietudes intelectuales y una 
gran capacidad de trabajo. En la primera reunión puso tal voluntad y tanto interés en el tema que 
decidí oírle más veces con el objetivo de encontrar el error en sus razonamientos. Y esa tarea, 
ciertamente, se me ha hecho muy difícil. No es fácil hallar el error. Algún resultado parcial, como la 
TIP, me resultó curioso e interesante, pero no puedo asegurar aún su validez. Tampoco sé cuál es 
el papel que juega en su demostración, porque no conozco la totalidad de la misma. Posiblemente 
cientos de páginas, con las dificultades que tiene de manejar el lenguaje matemático, frente a la 
sencilla prueba original de Cantor.\\
''\\
Palabras de José Manuel Méndez Pérez, catedrático de la ULL. Nos reunimos como unas 6 veces a lo largo de 3 años. Sus palabras son un ejemplo de como los resultados son difícilmente negables, y como la incredulidad, comprensible, impregna todas las reacciones. La TPI ``tan solo'' es una herramienta alternativa a las biyecciones, para comparar cardinalidades infinitas entre conjuntos. Y tengo una persona, con prestigio reconocido, diciendo claramente que le pareció una herramienta interesante. Y prometo que no fue un juez amable, fueron 6 reuniones en las que cuando encontraba fallos, los manifestaba sin piedad, hasta que los fui arreglando.

\noindent
Ahora debo añadir una anécdota, pero estas son MIS palabras, que con tiempo, puedo defender. La TPI completa se la he presentado a dos personas. Una de ellas, Pepe Mendez, como me deja llamarle, y otra una persona anónima de reddit. Ninguno de los dos, al acabar la definición de la TPI y su aplicación al caso $\mathbb{P(N)}$ vs $\mathbb{N}$, supo decir exactamente dónde estaba el fallo, pero ambos no estaban convencidos del todo, creían que el fallo DEBÍA estar en alguna parte. Lo gracioso, es que uno piensa que el error consiste en que estoy comparando, sin darme cuenta, $\aleph_{1}$ con $\aleph_{1}$. El otro piensa igual, solo que estoy comparando, sin darme cuenta, $\aleph_{0}$ con $\aleph_{0}$. Quitando que son tan inconfundibles que no saben bien cual escoger al ``suponer'' el fallo... NINGUNO de los dos dudó ni un segundo que estábamos hablando de conjuntos con el mismo cardinal.

\noindent
Puedo prometer tranquilamente que es un puzle tremendamente desafiante, porque tengo experiencia previa con expertos. Y se vuelve más interesante si hay más de uno presente, para escucharse entre ellos. Y no es lo único que he visto. La gente siempre resuelve esta anécdota, o parecidas, de la misma forma: `` ...a saber quienes serán esas personas...'' hasta que les toca a ellos tratar de resolver el puzle, y la actitud cambia de forma radical.

\noindent
Me he cruzado con gente que hasta ha llegado a citar mi propio trabajo, argumento por argumento... en lo que yo llamo, el ``cortocircuito''. Decirme que si fuese a la universidad, entendería mucho mejor ``mi propio trabajo''. Llegar a decirme a la cara que no tiene dudas que uno de mis antiguos trabajos es ABSOLUTAMENTE CORRECTO, al garantizarme la única pieza dudosa. El primer problema es que fue un trabajo descartado por ``crankery''. El segundo es que estaba de acuerdo en que si lo que decía no era cierto, el Teorema de Cantor era falso: por eso optó por esa opción, la otra desmontaba el teorema sin dudas.

\noindent
RESUMIENDO: Me he pasado 6 años chequeando, y revisando el trabajo. Cada punto es escogido por una persona diferente como incorrecto, de forma aleatoria, y a la vez, tengo varios matemáticos, que dicen que ese mismo punto es absolutamente correcto, pero escogen aleatoriamente otro punto, también aceptado por varias personas como obvio y trivial. O sea: sólido y correcto. Todos los puntos chequeados, pero de esa forma tan irregular. Por eso necesito varias personas.

\noindent
Para no tener estudios matemáticos, consigo desarrollar herramientas que les parecen interesantes a catedráticos. Y a más gente, pero son personas anónimas de internet. Uno decía tener 20 años de experiencia en teoría de conjuntos, y a una de mis herramientas la calificó de ``ingeniosa''.

\noindent
El trabajo ha alcanzado un estado de solidez tal, que ``confunde a matemáticos'' y les hace AFIRMAR lo mismo y lo contrario, en sus intentos desesperados de ``suponer'' el fallo. Porque claro, la observación final es IMPOSIBLE, así que el fallo debe estar en alguna parte. Pero ese camino les lleva a cuestiones muy extrañas, como afirmar lo que pretenden negar, sin darse cuenta.


\noindent
También se puede consultar con mi socio: Francisco Mario Cruz Almeida. Llamándole, en hora de Canarias y horario laboral, al 922 195005. Fue él quién me presentó a Pepe Méndez, después de un par de años presentándole resultados cada vez mejores. Tiene el trabajo un poco oxidado porque hace años que no lo mira, pero si os puede confirmar dos cosas. Una, que el, antes de conocerme, creía en la existencia de infinitos de diferentes tamaños. La segunda, que la semana pasada le ofrecí la ``biyección'' que lleva AÑOS pidiéndome... y me dijo: ``Es un buen punto de partida''.

\noindent
En 15 minutos puedo cambiarle la cara a un grupo de matemáticos sobre su fé en la inquebrantabilidad de la demostración. LA EXPERIENCIA me dice que hay dos grupos de matemáticos, cada uno usa una rama de la demostración y la creen perfecta y hermosa. Una, es la sencilla. Los que usan la segunda te lo dicen bien claro, que el fallo es que no has tenido en cuenta que la ``versión real'' es la que ellos dicen. Sin problemas... lo primero es pq al primer grupo NADIE les dice gran cosa ni se les puntualiza. O sea, es una demostración QUE CONVENCE A MATEMÁTICOS... y la de ellos tiene otros fallos. Se puede hacer una analogía rápida, para explicarles en que consiste, pero una vez acabado el experimento, como he tenido ya que desarrollar las herramientas necesarias, las puedo usar para indicarles una grieta GORDÍSIMA en esa técnica: lo que creían impredecible, es ABSOLUTAMENTE PREDECIBLE DE ANTEMANO. Pero claro, con mis alternativas a las biyecciones. Es un fenómeno numérico muy curioso... pero solo demuestra o apunta, a que las demostraciones son incorrectas, no a que el teorema sea falso. Por eso el experimento es más contundente.




	\chapter{Experimento vs paper}

\noindent
¿Por qué no está publicado como un paper? Pues porque no soy matemático. NO es que no esté publicado. Lo está en diferentes sitios, en diferentes formatos, diferentes partes y versiones.

\noindent
El experimento se necesita, para demostrar que el trabajo es digno de un esfuerzo, cuyo objetivo sea reescribirlo de una forma lo más estándar posible, de acuerdo a los protocolos matemáticos de la comunidad internacional.

\noindent
Mi estilo personal provoca unas fases de reacciones cíclicas... una serpiente que se muerde la cola:\\\\
I) Se leen cualquier cosa escrita de muy mala leche y deprisa (quién se lo lee), y luego afirman que TODO es una absoluta basura. A menudo usan palabras peores :D.\\\\
II) No sé cómo, pero alguien a veces se queda... y empiezo a preguntar frase por frase... y de repente, todo es mucho más claro, y ya no es tan basura. Incluso cambian de parecer, y empiezan a encontrar cosas que les parecen interesantes (varias, según la persona).\\\\
III) Encuentran su fallo aleatorio particular, se quedan contentos, y cortan comunicación. Pero cuando enseño ese mismo punto a otros matemáticos, lo consideran obvio y trivial, y me miran raro si les pido que lo confirmen dos veces... (se nota por el tono de escritura :D)\\\\
IV) Consigo que se queden hasta la conclusión final, dando okey de manera informal a cada punto. Pero una vez ven la conclusión final:\\
4a) O escogen una de las dos opciones del experimento (ya lo veremos). Lo cual sin saberlo, indica que tienen absoluta certeza que el teorema es falso, pero no lo saben.\\
4b) No pueden ver el fallo, lo admiten, pero suponen que debe existir. Y me aconsejan publicarlo para que otra persona me indique el fallo.\\\\

\noindent
Pero claro, si pudiese publicarlo en el formato y protocolos correctos, no necesitaría ni el experimento, ni la ayuda de nadie. Sin embargo, el experimento se usa para que las dos opciones de la cuarta fase, no hablen de ``mis palabras'', sino de las suyas propias. Y que contrasten lo que dicen, no con un desconocido sin estudios, sino con colegas, que dicen lo contrario que ellos, y ambos me dan la razón sin poder evitarlo.

\noindent
Y se cierra la serpiente que se muerde la cola: lo re-escribo, y mi estilo provoca otra vez la fase I, II, III...

\noindent
Si mi lenguaje y herramientas, poco ortodoxas, fuesen un problema REAL de rigor, Pepe Mendez me hubiese parado y hubiese indicado que ese era el fallo. El truco para entendernos es simplificar TANTO cada punto, que un matemático pueda decidir si lo considera correcto o no. Pero al no usar lenguaje ortodoxo, siempre les queda un mal sabor de boca al sacarles de su zona de confort.


	\chapter{El experimento}

\noindent
La exposición dura alrededor de 4 horas y media. Dependiendo de la valentía de los participantes se puede hacer en un solo dia, con descansos, empezando muy temprano. O en tres sesiones de tres días consecutivos. ES IMPORTANTE LA CERCANÍA TEMPORAL, pq es común que las personas olviden puntos que previamente han aceptado si se deja pasar tiempo.


\noindent
Las tres fases de la exposición son:


\noindent
\textbf{I) Conceptos previos y fenómeno numérico del ``empate cardinal'':} R-pares y D-pares; Relaciones no aplicación; Packs; Resolver un D-par; Naive CA Theorem(inyectividad es un caso particular); El rigor no siempre es necesario; Esquema general; $LCF_{2p}$ vs SNEIs; Propiedad Gamma de D-pares; Familias de D-pares; Relación FLJA abstracta; Relaciones $r_{\theta_{k}}$; Esquema final del fenómeno numérico. \textbf{(2 horas aprox)}.


\noindent
\textbf{II) TPI(Transferencia de Pares Ilimitada):} Idea generativa original(mención de cambio espectral de elementos a D-pares); Recordatorio CA $\equiv$ Inyectividad; Relaciones inyectivas imperfectas, WSP, NWSP y QRE; Ejemplo sencillo espectro finito (Dominio $\lll$ Origen); Puntualización, cambio de espectro de elementos a D-pares, animación de visualización de cardinalidad; Mención de algunos resultados del espectro finito; Definición de TPI (particularidades espectro infinito, no olvidarse ningún D-par, QRE inútil, subconjuntos anidados, intersección vacía); Aplicación a $\mathbb{P(N)}$ vs $\mathbb{N}$; Reducción del problema a una intersección infinita generalizada; Proceso de chopping infinito; Preguntas: ¿Se vacía NWSP?; ¿Podemos ignorar el cardinal de los subconjuntos?; ¿Podemos ignorar el resultado de vacío?;¿Podemos ignorar no ser capaces de mencionar un solo elemento del subconjunto de NWSP que sobrevive, si sobrevive algo? \textbf{(1 hora y media aprox)}.


\noindent
\textbf{III) DI(Diagonalización Inversa) variante pelea de colegio:} Explicación del juego cardinal; Descartar un miembro de $LCF_{2p}$ es descartar todo su universo, resolver un D-par es resolver toda su Familia; Lo que le pasa a un PACK, le pasa a todos los PACKs; Reducción del problema a una intersección infinita generalizada (observando un solo PACK); Proceso de chopping infinito; Preguntas: ¿Se vacían los PACKs?; ¿Podemos ignorar el cardinal de los subconjuntos?; ¿Podemos ignorar el resultado de vacío?;¿Podemos ignorar no ser capaces de mencionar un solo elemento del subconjunto de cada PACK que sobrevive, si sobrevive algo? \textbf{(45 mins aprox)}.


\noindent
Se crearán dos grupos de expertos de como mínimo 3 integrantes, cada grupo. Y dos jueces, como mínimo. Si hay publico se le podría hacer participar en las votaciones.


\noindent
El proceso será interactivo. Esto no será una exposición al uso. A menudo se realizarán preguntas, para ver si los integrantes de los grupos, consideran cada punto adecuado. Se tomará nota. La idea es que luego no puedan decir lo contrario, o al final, poder recordarles que cada punto aislado les parecía correcto a una cantidad ``interesante'' de miembros. No soy un alumno en un examen. No soy matemático. El proceso es un puzle... un desafío... un trabajo multidisciplinar entre la persona que expone y los integrantes de los grupos. EL BUEN JUICIO MATEMÁTICO, lo aportan los miembros de los grupos, no la persona que expone. Con la información aportada, deberán decidir si consideran cada punto ``razonablemente correcto''.


\noindent
La fase I es común a ambos grupos. La fase II se presenta SOLO a un grupo, y se apuntan sus respuestas a las preguntas. El otro grupo se aisla. La fase III se presenta al otro grupo, y se anotan sus respuestas a las preguntas. El primer grupo puede entrar a la siguiente fase si mantienen silencio y se van a la zona del público.

\noindent
Lo que YA ha sucedido, pero no en un experimento, sino en foros matemáticos a través de años de conversaciones, es que ambas técnicas, tanto la TPI como la DI, necesitan unas respuestas muy específicas a esas preguntas para poder decir que el cardinal de $\mathbb{P(N)}$ NO es mayor que el cardinal de $\mathbb{N}$. Respuestas que son complementarias. Cada grupo, para no negar el Teorema de Cantor, NEGARÁ las respuestas, que no puedo demostrar, y que se han preseleccionado sin ningún tipo de rubor ni vergüenza para poder acabar la construcción de las técnicas.  LO MALO, es que al negar las respuestas de una técnica, se le garantizan a la otra técnica, EXACTAMENTE las respuestas que necesita. Y si no las niegas, el teorema muere. El experimento consiste en la firmeza de la elección de los miembros de los grupos. No de la persona que expone, de los miembros de los grupos. Estarán ABSOLUTAMENTE seguros de sus elecciones. La última fase consiste en que cada grupo le lea al otro grupo sus respuestas, en voz alta.

\noindent
En realidad es un clásico caso de bifurcación, en el que ambos caminos llevan a la misma conclusión. No sabemos CUÁL es el verdadero, ni el correcto, pero como ambos apuntan a la misma conclusión final, se consideran todas las posibilidades cubiertas. Este final es para los indecisos, PERO que aceptan que esas preguntas solo tienen dos respuestas posibles: si el conjunto se vacía o no, pq el resto de preguntas son excusas para justificar nuestra elección en la primera.

\noindent
Si solo se expone, al ser tan radical el resultado... se suele responder con: ``No se dónde está el error... seguro que se ma ha escapado algo...''. Durante el proceso, cada paso, se ha considerado ``razonablemente correcto'' por los oyentes, no por la persona que expone. Matemáticos, no una persona sin estudios. MÍNIMO, este resultado final, se merece un estudio más profundo y hacerlo público para toda la comunidad, en búsqueda de gente mucho más experta que quiera estudiar el trabajo inquisitorialmente.

\noindent
Cada grupo es de más de un matemático, para que sus colegas le saquen de su cortocircuito personal. Para que le ayuden a esquivar el punto que su cerebro va a escoger de forma aleatoria, para intentar que todo siga encajando en el mundo de la teoría de conjuntos. Si solo estoy yo, puede el prestigio, y no se valora mi juicio.

\noindent
Los jueces tienen la misma labor. NO PUEDEN EMITIR VOTOS, ni decidir sobre cada punto. DEBEN jugar limpio, y sacar de su error a una persona, si pueden, en un punto concreto. DEBEN reprender a los miembros de los grupos que no ofrezcan respuestas honestas (se notan a la legua, pero en público, rodeado de colegas, no es lo mismo hacerse el despistado con cosas básicas, solo para reírse de mi falta de rigor formal). Y su tercera labor es evitar pérdidas de tiempo. Cortar de raíz cualquier intento de cambiar el flujo de la exposición, para demostrar lo ``supuestamente'' absurdo que es todo el proceso (antes de verlo entero). Discusiones sobre cambios de nombres o mejoras sobre la marcha sin tener visión general. Cada punto es correcto o no, no importa si es mejorable (muchas personas se equivocan con los cambios, por desconocer el panorama general, y los cambios de nombres son insustanciales)

\noindent
Cada punto NO SE VA A DEMOSTRAR de manera formal. Se va a exponer en una forma poco ortodoxa. Posiblemente poniendo nombres diferentes a conceptos conocidos... llevo AÑOS pidiendo ayuda para arreglar eso, pero el experimento es para demostrar que el trabajo se merece esa ayuda. Cada voto positivo significa: ``Estoy de acuerdo con el punto, me parece razonable, y si alguien quiere, al acabar hoy, le puedo hacer una defensa formal de dicho punto, aunque la persona que expone no la ofrezca''. Se anotarán, en cada punto, los votos favorables, y si se tercia, los del público. Sólo favorables.

\noindent
Un inciso. Yo soy un matemático intuitivo. Tengo cierto instinto para decidir qué puede ser demostrable, de manera formal y rigurosa, fácilmente, y que no, aunque soy incapaz de hacerlo personalmente. Me equivoco, pero a menudo no. PARA EVITAR actitudes deshonestas, pq recordemos que esto es un puzle, un trabajo en equipo, CONOZCO algunas demostraciones de algunos puntos. A MI ME BASTA con haber consultado durante años a matemáticos y que me confirmasen que cada uno es correcto, excepto las respuestas a las preguntas :D. Pero a veces me he encontrado con gente que se hace la ignorante adrede, CONOCEN la respuesta, la ocultan, pero simplemente tratan de jugar la baza del rigor para poder escaparse. NO VOY A DECIR QUE PUNTOS son los que conozco su demostración formal. Si alguien intenta ser deshonesto, no creo que le agrade que una persona que ni siquiera tiene primero de matemáticas le enseñe una demostración ultra sencilla que el o ella decían que no eran capaces de ver. Esto se avisa al inicio. De resto no pasa nada, las herramientas son demasiado poco ortodoxas. Es normal que algunas cosas no se vean a la primera. He procurado aportar lo suficiente, en el grado de simpleza máximo que puedo alcanzar, para que se pueda decidir sobre ellas, o yo aporte argumentos en su defensa, alternativos, si es necesario. La idea es que una persona, o más, de cada grupo, acepte un nivel de desarrollo, a partir del cuál ellos puedan crear, con tiempo, una demostración formal de ese punto, y con su voto, avisan al resto de miembros del equipo. FUERA del experimento, con más tiempo y calma, se puede analizar si realmente se podía.


\noindent
Una vez acabado, OBSERVADO TODO EL PROCESO, tanto los integrantes de los grupos deben decidir si merece la pena DECIR EN PÚBLICO, que este puzle merece un estudio mucho más profundo, recomendando su estudio a gente mucho más experta si fuese necesario. Incluidos los jueces.

\noindent
En caso de tener éxito, se puede hablar sobre expandir la exposición a otros días. CONSEGUIR que una sala entera llena de matemáticos DUDE del Teorema de Cantor, ya es un logro más que notable. Igual algunas personas abandonan la duda y comienzan a hacerse más preguntas, para las que tengo respuestas (y señalo puertas), pero en un entorno más desenfadado y de colaboración, expandiendo las charlas publicadas en vídeo.

\noindent
1.- Si ambos conjuntos tienen el mismo cardinal, ¿Qué ha fallado en las demostraciones? Paradojas Híbridas, ``esquemas'' de axiomas ZF incompletos, pensar que la diferencia entre lo ``enumerable'' y lo contínuo era su cardinalidad por culpa de Cantor. Definiciones imperfectas.\\\\
2.- Fallos en las demostraciones del teorema, fallo de la doble contradicción(reproducible entre conjuntos del mismo cardinal), fallo de la diagonalización (Predecir TODA posible diagonalización, de antemano). TODO elemento de $\mathbb{P(N)}$, es el subconjunto mágico de Cantor en algún intento de función biyectiva. Cuando la diagonalización genera un subconjunto ya cubierto en otra zona de la función por partes.\\\\
3.- LA TPI, en realidad, es tremendamente parecida a una función inyectiva por partes: similitudes y diferencias. Cantor-Bernstein-Schröder cogido con pinzas. Espíritu de una idea, versus, su definición rigurosa: biyección e igualdad cardinal.\\\\
4.- LAS CONSTRUCCIONES LJA (Esto ya requiere varios días). Un patrón $\pi$-recursivo, común a muchas biyecciones famosas, aplicadas a casos muy diversos: más allá de $\mathbb{P(N)}$ ( Reales, conjunto de Cantor...). Ir más allá de $\aleph_{1}$. Ir más allá de $\aleph$ con subíndices naturales. La alternativa al uso de potencias de números primos.\\\\
5.- Chequear y ordenar TODO mi material: redactar lo que se pueda de manera formal. Publicarlo.\\\\
6.- Investigaciones futuras. Aprender a definir las Paradojas Híbridas, no solo detectar casos particulares. Aprender a evitar su uso futuro. Encontrar TODOS los teoremas mal demostrados o falsos. Crear nuevas demostraciones. Redefinir ZF. Recuperar el prestigio del constructivismo, recuperar a los matemáticos intuitivos que solo saben construir, pero no demostrar. Gödel y Turing cometieron un fallo similar a Cantor? Usaron paradojas híbridas para demostrar teoremas falsos?\\\\
7.- SOLUCIÓN DEFINITIVA de la hipótesis del contínuo. Repasar antiguos fenómenos conocidos, que eran considerados ``contra intuitivos y curiosos'', pero no pistas sobre que el teorema, en realidad, fallaba. P.e: entre cualesquiera dos números Irracionales diferentes, SIEMPRE hay una cantidad infinita de números Racionales. Las limitaciones de los cortes de Dedekin. La propia locura de resultados que ofrecía la hipótesis.¿Contra intuitivo significa a veces erróneo?\\\\
8.- Grabar a fuego la lección para el futuro: un simple detalle, tremenda e increíblemente sutil, puede llevar a la construcción de un árbol entero de teoremas falsos, aceptados como correctos durante generaciones, sobre todo si mitificamos la ``supuesta'' perfección de las matemáticas y tratamos a los teoremas como dioses sobre los cuales no cabe crítica alguna. La verdad no se rompe por mucho que la pongas a prueba.\\\\
9.- Cantor, el hacker más grande de la historia, con el mayor logro posible: hackear el sistema de reglas más seguro diseñado por el ser humano. Enemigo, maestro y colega. Un trabajo transgeneracional, creado en dos fases. A la caza y captura de las paradojas híbridas. Confesar mi búsqueda vital de lo que Cantor creó, y yo siempre fui incapaz de crear.\\\\
10.- Usar super computación, para reducir el record máximo de densidad calculada de $(LCF_{1} \cup LCF_{2p})$ dentro de $LCF$. Miembros útiles / Miembros totales. Programar una web dónde la gente pueda jugar con las CLJAs. Definir las suyas propias. Diseñar CLJAs compuestas de forma gráfica. Crear módulos y librerías para software muy usado. Programar las CLJAS del punto 4. ``Enumerar'' TODOS los ordinales, programar esa CLJA.\\\\
11.- Forma normal de Chomsky y CLJAs, mejorando su precisión para lenguajes (conjuntos) más complejos y precisos. Normas de equivalencias entre gramáticas y CLJAs compuestas. Nomenclatura de composiciones recursivas e infinitas de CLJAs: ciclos (no círculos, sino espirales de tubo, para evitar tener más de una fuente). Nomenclatura de conjuntos LCF con trillones de paréntesis:\\
(-$<$número$>$-(... lambdas, caminos finitos, etc ...)-$<$número$>$-).





	\chapter{Condiciones}

\noindent
1.- Una copia en soporte físico de todo el material grabado. Derecho a poder disponer de todo el audio e imágenes como me plazca, y distribuirlo como me plazca. Pasado un tiempo razonable, un mes, por ejemplo, en el cual se puede disfrutar de exclusividad completa de la distribución del material, yo podré hacer lo mismo por mi cuenta, incluso copiando el material montado, distribuido en diversas plataformas. Simplemente quiero poder asegurarme que SIEMPRE va a estar disponible. O protegerme de montajes malintencionados. Mientras siga de acceso libre no tengo motivos para distribuir mi copia del material, pero sigue quedando bajo mi total discreción decidir lo que hacer, pasado el tiempo de exclusividad. Eso no os quitará el derecho a seguir distribuyéndolo por vuestra cuenta. Quién sea que monte el experimento puede usarlo como una plataforma de marketing académico y conservar los ingresos que le reporte dicha distribución, siempre y cuando lo disponga de acceso público, como compensación por montar el experimento. Eso no implica una exclusividad eterna ni apropiación exclusiva del material, ni impedirme generar mis propios ingresos con el material. Repito, deseo TOTAL poder de distribución y conservación del material, inclusive con fines económicos.

\noindent
2.- El compromiso se ciñe a la realización del experimento y su exposición pública. En ningún momento acepto ningún lazo de unión o compromiso con ninguna institución, ni cedo la autoría, ni derechos de explotación, o decisión, más allá de poder distribuir el material, un tiempo en exclusiva, y luego en paralelo. Tampoco concedo ningún poder a nadie de poder negociar en mi nombre con ninguna institución.

\noindent
3.- La Universidad de La Laguna tiene un salón de actos con dispositivos de grabación de vídeo. No creo que sea la elección, pero una petición formal a la delegación de alumnos o a la vicedecana que ya me conoce, podría ser exitosa y concederse el permiso. En caso de querer montarlo en otro lugar, requeriré de billetes de avión, alojamiento en hotel y dinero diario para comer y desplazarme por la zona. Calculo unos 200 euros al día, que se me ingresarán de antemano, aparte del alojamiento y los billetes de avión ( con equipaje de mano y dos maletas por si acaso). Casi cualquier hotel me vale, mientras tenga habitación individual con baño y un lugar en recepción con cierta seguridad donde dejar mi portátil.

\noindent
4.- En caso de éxito del experimento, y que os impresione lo suficiente, el resto se puede negociar. Mi idea siempre ha sido pedir colaboración entre diversas instituciones, porque no creo que una sola tenga los recursos necesarios para el proyecto que tengo en la cabeza. Aparte que me encantaría poder crear un equipo internacional, para poder obtener las mejores soluciones posibles y no arreglarlo todo con un simple parche medio absurdo. Desde mi ignorancia, la paradoja de Russell se arregló (DESDE MI IGNORANCIA) con un simple juego de palabras, y resultaba que ese fenómeno impregnaba más cosas y no bastaba con esconderlo debajo de la alfombra.

\noindent
5.- NO se guardará en secreto ni haré exposiciones en privado, con la idea de sacar un paper dentro de uno o tres años. PRIMERO se hará público el experimento lo antes posible, si es que no se emite en directo en alguna plataforma. Y luego hablaremos de proyectos académicos. El prestigio del descubrimiento lo deseo en exclusiva para mi y para mi socio. El prestigio del paper, o papers, que sería uno de los productos finales del proyecto, o los diversos futuros descubrimientos, no me importa compartirlo con más gente. Hay tesoros académicos para todos si conseguimos desarrollar una relación de confianza y colaboración. A mi solo me interesa la exclusividad de la refutación del Teorema de Cantor, la solución definitiva de la hipótesis del contínuo, el postulado de la  hipótesis de la existencia de las Paradojas Híbridas (no su solución) y el desarrollo de las primeras Construcciones LJA. El resto, se puede compartir sin problemas, e incluso ceder la primera firma, todo negociable. Deseo un trabajo, pero una carrera académica me da bastante igual. Solo esos puntos, por la cantidad de sufrimiento y tiempo que me han costado, no son negociables en absoluto. Mi modelo de negocio es recuperar todo estos años sin ingresos a través de premios internacionales y con el proyecto de investigación.

\noindent
Indico esto para que se conozcan mis motivaciones. La elaboración de un contrato simple y claro, lo vería como un acto de buena fé y un buen comienzo. No creo pedir demasiado, dado que la inversión es muy pequeña y estamos hablando de un descubrimiento, quizás, de una talla superior a un problema del milenio. Sin lugar a dudas quedará marcado a fuego en la historia de las matemáticas mientras la civilización humana exista.
 

	
	%\part{English}
	%\chapter{Introducción}

\noindent
La propuesta consiste básicamente en realizar un experimento, barato, rápido y sencillo, de una duración aproximada de 4 horas y media, capaz de demostrar que el cardinal de $\mathbb{P(N)}$ NO es mayor que el cardinal de $\mathbb{N}$. No sólo con lo que todo ello implica, sino como una puerta de salida del paraíso que creó Cantor, que lleva a un mundo de conceptos nuevos y resultados increíbles, y muy necesarios. Algunos solo necesitan ser recopilados de forma adecuada, y otros requieren un equipo de investigación.

\noindent
¿Por qué un experimento y no un paper? Por muy diversas pero poderosas razones. La idea básica, y muy resumida, es que es difícil creerlo hasta que no se ve. En diversos aspectos. Se puede hablar durante dias, semanas o meses, cuando una exposición de 4 horas y media puede cerrar todos los debates. La idea del experimento es sacar al autor original de la ecuación, y que sean otros matemáticos quienes se enfrenten al fenómeno matemático, teniendo que lidiar con sus propias palabras y sus consecuencias, que encima, ellos mismos van a elegir.

\noindent
El autor original carece de prestigio, pero no de razón ni resultados. Eso lleva a muy variopintas reacciones... que hay que ver para creer. La idea base es reunir a un grupo de gente dispuesta a retar su mente y sus conocimientos preestablecidos, y observar sus reacciones al lidiar con un puzzle imposible de resolver, sin negar por el camino el teorema. Gente dispuesta a hacer público el material grabado en vídeo, y usar la distribución de ese vídeo para conseguir convenios con varias universidades, para obtener los recursos y personal adecuados para transformar dicho experimento, en un paper formal aceptado por toda la comunidad internacional, en el cual ya se recopilan TODOS los resultados y no solo un subconjunto de ellos (que es lo que sería el experimento).

\noindent
A los participantes se les ofrece a cambio, ser partícipes de un momento histórico, que será referenciado durante siglos. Y si se consiguen los recursos, ser considerados para una oferta de un puesto en el equipo de trabajo futuro. Ese equipo creará un paper que será considerado un básico en la formación de cualquier matemático en las futuras generaciones. No solo por sustituir gran parte de la actual teoría de conjuntos con muchos conceptos nuevos; no solo por cambiar paradigmas establecidos; sino también por la necesidad de encontrar nuevas demostraciones a viejos teoremas básicos y crear axiomas nuevos.

\noindent
Cualquier institución que se relacione con la creación y distribución pública de dicho experimento, tendrá una de las mejores campañas de marketing que se pueden soñar para una institución de ingeniería, ciencia o investigación. A cambio del ridículo precio de montar el experimento y un riesgo muy reducido. Ni siquiera hace falta darle mucha pompa, sino simplemente venderlo como un desafío informal, las reacciones de los matemáticos presentes hablarán por sí solas.

\newpage
\noindent
¿Qué preguntas pretende responder está presentación?:\\\\
¿Qué garantías se ofrecen para decidir si apoyar la realización del experimento?\\
¿Por qué no escribirlo en un simple paper y publicarlo?\\
¿En qué consistiría el experimento? Así se puede valorar su coste\\
Condiciones a tener en cuenta si el proyecto resulta interesante.

	%\input{./Peticion.tex}
	%\input{./Fases.tex}
	%\input{./Usos.tex}
	%\input{./Garantias.tex}
\end{document}