\chapter{El experimento}

\noindent
La exposición dura alrededor de 4 horas y media. Dependiendo de la valentía de los participantes se puede hacer en un solo dia, con descansos, empezando muy temprano. O en tres sesiones de tres días consecutivos. ES IMPORTANTE LA CERCANÍA TEMPORAL, pq es común que las personas olviden puntos que previamente han aceptado si se deja pasar tiempo.


\noindent
Las tres fases de la exposición son:


\noindent
\textbf{I) Conceptos previos y fenómeno numérico del ``empate cardinal'':} R-pares y D-pares; Relaciones no aplicación; Packs; Resolver un D-par; Naive CA Theorem(inyectividad es un caso particular); El rigor no siempre es necesario; Esquema general; $LCF_{2p}$ vs SNEIs; Propiedad Gamma de D-pares; Familias de D-pares; Relación FLJA abstracta; Relaciones $r_{\theta_{k}}$; Esquema final del fenómeno numérico. \textbf{(2 horas aprox)}.


\noindent
\textbf{II) TPI(Transferencia de Pares Ilimitada):} Idea generativa original(mención de cambio espectral de elementos a D-pares); Recordatorio CA $\equiv$ Inyectividad; Relaciones inyectivas imperfectas, WSP, NWSP y QRE; Ejemplo sencillo espectro finito (Dominio $\lll$ Origen); Puntualización, cambio de espectro de elementos a D-pares, animación de visualización de cardinalidad; Mención de algunos resultados del espectro finito; Definición de TPI (particularidades espectro infinito, no olvidarse ningún D-par, QRE inútil, subconjuntos anidados, intersección vacía); Aplicación a $\mathbb{P(N)}$ vs $\mathbb{N}$; Reducción del problema a una intersección infinita generalizada; Proceso de chopping infinito; Preguntas: ¿Se vacía NWSP?; ¿Podemos ignorar el cardinal de los subconjuntos?; ¿Podemos ignorar el resultado de vacío?;¿Podemos ignorar no ser capaces de mencionar un solo elemento del subconjunto de NWSP que sobrevive, si sobrevive algo? \textbf{(1 hora y media aprox)}.


\noindent
\textbf{III) DI(Diagonalización Inversa) variante pelea de colegio:} Explicación del juego cardinal; Descartar un miembro de $LCF_{2p}$ es descartar todo su universo, resolver un D-par es resolver toda su Familia; Lo que le pasa a un PACK, le pasa a todos los PACKs; Reducción del problema a una intersección infinita generalizada (observando un solo PACK); Proceso de chopping infinito; Preguntas: ¿Se vacían los PACKs?; ¿Podemos ignorar el cardinal de los subconjuntos?; ¿Podemos ignorar el resultado de vacío?;¿Podemos ignorar no ser capaces de mencionar un solo elemento del subconjunto de cada PACK que sobrevive, si sobrevive algo? \textbf{(45 mins aprox)}.


\noindent
Se crearán dos grupos de expertos de como mínimo 3 integrantes, cada grupo. Y dos jueces, como mínimo. Si hay publico se le podría hacer participar en las votaciones.


\noindent
El proceso será interactivo. Esto no será una exposición al uso. A menudo se realizarán preguntas, para ver si los integrantes de los grupos, consideran cada punto adecuado. Se tomará nota. La idea es que luego no puedan decir lo contrario, o al final, poder recordarles que cada punto aislado les parecía correcto a una cantidad ``interesante'' de miembros. No soy un alumno en un examen. No soy matemático. El proceso es un puzle... un desafío... un trabajo multidisciplinar entre la persona que expone y los integrantes de los grupos. EL BUEN JUICIO MATEMÁTICO, lo aportan los miembros de los grupos, no la persona que expone. Con la información aportada, deberán decidir si consideran cada punto ``razonablemente correcto''.


\noindent
La fase I es común a ambos grupos. La fase II se presenta SOLO a un grupo, y se apuntan sus respuestas a las preguntas. El otro grupo se aisla. La fase III se presenta al otro grupo, y se anotan sus respuestas a las preguntas. El primer grupo puede entrar a la siguiente fase si mantienen silencio y se van a la zona del público.

\noindent
Lo que YA ha sucedido, pero no en un experimento, sino en foros matemáticos a través de años de conversaciones, es que ambas técnicas, tanto la TPI como la DI, necesitan unas respuestas muy específicas a esas preguntas para poder decir que el cardinal de $\mathbb{P(N)}$ NO es mayor que el cardinal de $\mathbb{N}$. Respuestas que son complementarias. Cada grupo, para no negar el Teorema de Cantor, NEGARÁ las respuestas, que no puedo demostrar, y que se han preseleccionado sin ningún tipo de rubor ni vergüenza para poder acabar la construcción de las técnicas.  LO MALO, es que al negar las respuestas de una técnica, se le garantizan a la otra técnica, EXACTAMENTE las respuestas que necesita. Y si no las niegas, el teorema muere. El experimento consiste en la firmeza de la elección de los miembros de los grupos. No de la persona que expone, de los miembros de los grupos. Estarán ABSOLUTAMENTE seguros de sus elecciones. La última fase consiste en que cada grupo le lea al otro grupo sus respuestas, en voz alta.

\noindent
En realidad es un clásico caso de bifurcación, en el que ambos caminos llevan a la misma conclusión. No sabemos CUÁL es el verdadero, ni el correcto, pero como ambos apuntan a la misma conclusión final, se consideran todas las posibilidades cubiertas. Este final es para los indecisos, PERO que aceptan que esas preguntas solo tienen dos respuestas posibles: si el conjunto se vacía o no, pq el resto de preguntas son excusas para justificar nuestra elección en la primera.

\noindent
Si solo se expone, al ser tan radical el resultado... se suele responder con: ``No se dónde está el error... seguro que se ma ha escapado algo...''. Durante el proceso, cada paso, se ha considerado ``razonablemente correcto'' por los oyentes, no por la persona que expone. Matemáticos, no una persona sin estudios. MÍNIMO, este resultado final, se merece un estudio más profundo y hacerlo público para toda la comunidad, en búsqueda de gente mucho más experta que quiera estudiar el trabajo inquisitorialmente.

\noindent
Cada grupo es de más de un matemático, para que sus colegas le saquen de su cortocircuito personal. Para que le ayuden a esquivar el punto que su cerebro va a escoger de forma aleatoria, para intentar que todo siga encajando en el mundo de la teoría de conjuntos. Si solo estoy yo, puede el prestigio, y no se valora mi juicio.

\noindent
Los jueces tienen la misma labor. NO PUEDEN EMITIR VOTOS, ni decidir sobre cada punto. DEBEN jugar limpio, y sacar de su error a una persona, si pueden, en un punto concreto. DEBEN reprender a los miembros de los grupos que no ofrezcan respuestas honestas (se notan a la legua, pero en público, rodeado de colegas, no es lo mismo hacerse el despistado con cosas básicas, solo para reírse de mi falta de rigor formal). Y su tercera labor es evitar pérdidas de tiempo. Cortar de raíz cualquier intento de cambiar el flujo de la exposición, para demostrar lo ``supuestamente'' absurdo que es todo el proceso (antes de verlo entero). Discusiones sobre cambios de nombres o mejoras sobre la marcha sin tener visión general. Cada punto es correcto o no, no importa si es mejorable (muchas personas se equivocan con los cambios, por desconocer el panorama general, y los cambios de nombres son insustanciales)

\noindent
Cada punto NO SE VA A DEMOSTRAR de manera formal. Se va a exponer en una forma poco ortodoxa. Posiblemente poniendo nombres diferentes a conceptos conocidos... llevo AÑOS pidiendo ayuda para arreglar eso, pero el experimento es para demostrar que el trabajo se merece esa ayuda. Cada voto positivo significa: ``Estoy de acuerdo con el punto, me parece razonable, y si alguien quiere, al acabar hoy, le puedo hacer una defensa formal de dicho punto, aunque la persona que expone no la ofrezca''. Se anotarán, en cada punto, los votos favorables, y si se tercia, los del público. Sólo favorables.

\noindent
Un inciso. Yo soy un matemático intuitivo. Tengo cierto instinto para decidir qué puede ser demostrable, de manera formal y rigurosa, fácilmente, y que no, aunque soy incapaz de hacerlo personalmente. Me equivoco, pero a menudo no. PARA EVITAR actitudes deshonestas, pq recordemos que esto es un puzle, un trabajo en equipo, CONOZCO algunas demostraciones de algunos puntos. A MI ME BASTA con haber consultado durante años a matemáticos y que me confirmasen que cada uno es correcto, excepto las respuestas a las preguntas :D. Pero a veces me he encontrado con gente que se hace la ignorante adrede, CONOCEN la respuesta, la ocultan, pero simplemente tratan de jugar la baza del rigor para poder escaparse. NO VOY A DECIR QUE PUNTOS son los que conozco su demostración formal. Si alguien intenta ser deshonesto, no creo que le agrade que una persona que ni siquiera tiene primero de matemáticas le enseñe una demostración ultra sencilla que el o ella decían que no eran capaces de ver. Esto se avisa al inicio. De resto no pasa nada, las herramientas son demasiado poco ortodoxas. Es normal que algunas cosas no se vean a la primera. He procurado aportar lo suficiente, en el grado de simpleza máximo que puedo alcanzar, para que se pueda decidir sobre ellas, o yo aporte argumentos en su defensa, alternativos, si es necesario. La idea es que una persona, o más, de cada grupo, acepte un nivel de desarrollo, a partir del cuál ellos puedan crear, con tiempo, una demostración formal de ese punto, y con su voto, avisan al resto de miembros del equipo. FUERA del experimento, con más tiempo y calma, se puede analizar si realmente se podía.


\noindent
Una vez acabado, OBSERVADO TODO EL PROCESO, tanto los integrantes de los grupos deben decidir si merece la pena DECIR EN PÚBLICO, que este puzle merece un estudio mucho más profundo, recomendando su estudio a gente mucho más experta si fuese necesario. Incluidos los jueces.

\noindent
En caso de tener éxito, se puede hablar sobre expandir la exposición a otros días. CONSEGUIR que una sala entera llena de matemáticos DUDE del Teorema de Cantor, ya es un logro más que notable. Igual algunas personas abandonan la duda y comienzan a hacerse más preguntas, para las que tengo respuestas (y señalo puertas), pero en un entorno más desenfadado y de colaboración, expandiendo las charlas publicadas en vídeo.

\noindent
1.- Si ambos conjuntos tienen el mismo cardinal, ¿Qué ha fallado en las demostraciones? Paradojas Híbridas, ``esquemas'' de axiomas ZF incompletos, pensar que la diferencia entre lo ``enumerable'' y lo contínuo era su cardinalidad por culpa de Cantor. Definiciones imperfectas.\\\\
2.- Fallos en las demostraciones del teorema, fallo de la doble contradicción(reproducible entre conjuntos del mismo cardinal), fallo de la diagonalización (Predecir TODA posible diagonalización, de antemano). TODO elemento de $\mathbb{P(N)}$, es el subconjunto mágico de Cantor en algún intento de función biyectiva. Cuando la diagonalización genera un subconjunto ya cubierto en otra zona de la función por partes.\\\\
3.- LA TPI, en realidad, es tremendamente parecida a una función inyectiva por partes: similitudes y diferencias. Cantor-Bernstein-Schröder cogido con pinzas. Espíritu de una idea, versus, su definición rigurosa: biyección e igualdad cardinal.\\\\
4.- LAS CONSTRUCCIONES LJA (Esto ya requiere varios días). Un patrón $\pi$-recursivo, común a muchas biyecciones famosas, aplicadas a casos muy diversos: más allá de $\mathbb{P(N)}$ ( Reales, conjunto de Cantor...). Ir más allá de $\aleph_{1}$. Ir más allá de $\aleph$ con subíndices naturales. La alternativa al uso de potencias de números primos.\\\\
5.- Chequear y ordenar TODO mi material: redactar lo que se pueda de manera formal. Publicarlo.\\\\
6.- Investigaciones futuras. Aprender a definir las Paradojas Híbridas, no solo detectar casos particulares. Aprender a evitar su uso futuro. Encontrar TODOS los teoremas mal demostrados o falsos. Crear nuevas demostraciones. Redefinir ZF. Recuperar el prestigio del constructivismo, recuperar a los matemáticos intuitivos que solo saben construir, pero no demostrar. Gödel y Turing cometieron un fallo similar a Cantor? Usaron paradojas híbridas para demostrar teoremas falsos?\\\\
7.- SOLUCIÓN DEFINITIVA de la hipótesis del contínuo. Repasar antiguos fenómenos conocidos, que eran considerados ``contra intuitivos y curiosos'', pero no pistas sobre que el teorema, en realidad, fallaba. P.e: entre cualesquiera dos números Irracionales diferentes, SIEMPRE hay una cantidad infinita de números Racionales. Las limitaciones de los cortes de Dedekin. La propia locura de resultados que ofrecía la hipótesis.¿Contra intuitivo significa a veces erróneo?\\\\
8.- Grabar a fuego la lección para el futuro: un simple detalle, tremenda e increíblemente sutil, puede llevar a la construcción de un árbol entero de teoremas falsos, aceptados como correctos durante generaciones, sobre todo si mitificamos la ``supuesta'' perfección de las matemáticas y tratamos a los teoremas como dioses sobre los cuales no cabe crítica alguna. La verdad no se rompe por mucho que la pongas a prueba.\\\\
9.- Cantor, el hacker más grande de la historia, con el mayor logro posible: hackear el sistema de reglas más seguro diseñado por el ser humano. Enemigo, maestro y colega. Un trabajo transgeneracional, creado en dos fases. A la caza y captura de las paradojas híbridas. Confesar mi búsqueda vital de lo que Cantor creó, y yo siempre fui incapaz de crear.\\\\
10.- Usar super computación, para reducir el record máximo de densidad calculada de $(LCF_{1} \cup LCF_{2p})$ dentro de $LCF$. Miembros útiles / Miembros totales. Programar una web dónde la gente pueda jugar con las CLJAs. Definir las suyas propias. Diseñar CLJAs compuestas de forma gráfica. Crear módulos y librerías para software muy usado. Programar las CLJAS del punto 4. ``Enumerar'' TODOS los ordinales, programar esa CLJA.\\\\
11.- Forma normal de Chomsky y CLJAs, mejorando su precisión para lenguajes (conjuntos) más complejos y precisos. Normas de equivalencias entre gramáticas y CLJAs compuestas. Nomenclatura de composiciones recursivas e infinitas de CLJAs: ciclos (no círculos, sino espirales de tubo, para evitar tener más de una fuente). Nomenclatura de conjuntos LCF con trillones de paréntesis:\\
(-$<$número$>$-(... lambdas, caminos finitos, etc ...)-$<$número$>$-).




