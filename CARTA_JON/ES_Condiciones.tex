\chapter{Condiciones}

\noindent
1.- Una copia en soporte físico de todo el material grabado. Derecho a poder disponer de todo el audio e imágenes como me plazca, y distribuirlo como me plazca. Pasado un tiempo razonable, un mes, por ejemplo, en el cual se puede disfrutar de exclusividad completa de la distribución del material, yo podré hacer lo mismo por mi cuenta, incluso copiando el material montado, distribuido en diversas plataformas. Simplemente quiero poder asegurarme que SIEMPRE va a estar disponible. O protegerme de montajes malintencionados. Mientras siga de acceso libre no tengo motivos para distribuir mi copia del material, pero sigue quedando bajo mi total discreción decidir lo que hacer, pasado el tiempo de exclusividad. Eso no os quitará el derecho a seguir distribuyéndolo por vuestra cuenta. Quién sea que monte el experimento puede usarlo como una plataforma de marketing académico y conservar los ingresos que le reporte dicha distribución, siempre y cuando lo disponga de acceso público, como compensación por montar el experimento. Eso no implica una exclusividad eterna ni apropiación exclusiva del material, ni impedirme generar mis propios ingresos con el material. Repito, deseo TOTAL poder de distribución y conservación del material, inclusive con fines económicos.

\noindent
2.- El compromiso se ciñe a la realización del experimento y su exposición pública. En ningún momento acepto ningún lazo de unión o compromiso con ninguna institución, ni cedo la autoría, ni derechos de explotación, o decisión, más allá de poder distribuir el material, un tiempo en exclusiva, y luego en paralelo. Tampoco concedo ningún poder a nadie de poder negociar en mi nombre con ninguna institución.

\noindent
3.- La Universidad de La Laguna tiene un salón de actos con dispositivos de grabación de vídeo. No creo que sea la elección, pero una petición formal a la delegación de alumnos o a la vicedecana que ya me conoce, podría ser exitosa y concederse el permiso. En caso de querer montarlo en otro lugar, requeriré de billetes de avión, alojamiento en hotel y dinero diario para comer y desplazarme por la zona. Calculo unos 200 euros al día, que se me ingresarán de antemano, aparte del alojamiento y los billetes de avión ( con equipaje de mano y dos maletas por si acaso). Casi cualquier hotel me vale, mientras tenga habitación individual con baño y un lugar en recepción con cierta seguridad donde dejar mi portátil.

\noindent
4.- En caso de éxito del experimento, y que os impresione lo suficiente, el resto se puede negociar. Mi idea siempre ha sido pedir colaboración entre diversas instituciones, porque no creo que una sola tenga los recursos necesarios para el proyecto que tengo en la cabeza. Aparte que me encantaría poder crear un equipo internacional, para poder obtener las mejores soluciones posibles y no arreglarlo todo con un simple parche medio absurdo. Desde mi ignorancia, la paradoja de Russell se arregló (DESDE MI IGNORANCIA) con un simple juego de palabras, y resultaba que ese fenómeno impregnaba más cosas y no bastaba con esconderlo debajo de la alfombra.

\noindent
5.- NO se guardará en secreto ni haré exposiciones en privado, con la idea de sacar un paper dentro de uno o tres años. PRIMERO se hará público el experimento lo antes posible, si es que no se emite en directo en alguna plataforma. Y luego hablaremos de proyectos académicos. El prestigio del descubrimiento lo deseo en exclusiva para mi y para mi socio. El prestigio del paper, o papers, que sería uno de los productos finales del proyecto, o los diversos futuros descubrimientos, no me importa compartirlo con más gente. Hay tesoros académicos para todos si conseguimos desarrollar una relación de confianza y colaboración. A mi solo me interesa la exclusividad de la refutación del Teorema de Cantor, la solución definitiva de la hipótesis del contínuo, el postulado de la  hipótesis de la existencia de las Paradojas Híbridas (no su solución) y el desarrollo de las primeras Construcciones LJA. El resto, se puede compartir sin problemas, e incluso ceder la primera firma, todo negociable. Deseo un trabajo, pero una carrera académica me da bastante igual. Solo esos puntos, por la cantidad de sufrimiento y tiempo que me han costado, no son negociables en absoluto. Mi modelo de negocio es recuperar todo estos años sin ingresos a través de premios internacionales y con el proyecto de investigación.

\noindent
Indico esto para que se conozcan mis motivaciones. La elaboración de un contrato simple y claro, lo vería como un acto de buena fé y un buen comienzo. No creo pedir demasiado, dado que la inversión es muy pequeña y estamos hablando de un descubrimiento, quizás, de una talla superior a un problema del milenio. Sin lugar a dudas quedará marcado a fuego en la historia de las matemáticas mientras la civilización humana exista.
 
