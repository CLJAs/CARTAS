\chapter{Garantías}

\noindent
La primera garantía es que el experimento YA ha tenido éxito, pero en foros matemáticos y reuniones privadas. La idea es juntar a personas que muy probablemente van a reaccionar de la misma manera y hacerles escucharse unos a otros.

\noindent
En una conversación privada se pueden ofrecer decenas de anécdotas de todo tipo e índole con gente de diversos niveles. Desde catedráticos a estudiantes, a gente que supuestamente ha hecho su tésis con Terence Tao, físicos, ingenieros... No sirven de mucho, pq nadie cree que sean realidad. Es UNA de las razones por las cuales se necesita dejar testimonio público.

\noindent
Se pueden ofrecer resultados, y dejar que las matemáticas hablen, pero claro, eso necesita tiempo. Hay resultados más sencillos, otros más complejos, algunos curiosos, y otros, como descubrí hace unos meses, decisivos y rotundos. La SEGUNDA razón es evitar que la gente salga huyendo ante la imposibilidad de negar un resultado. Tantas reacciones diferentes que contar... una muy común es decir: ``No soy capaz de encontrar el fallo en la exposición, pero alguien más experto en el campo SEGURO que lo encuentra''. Normalmente no encontrar fallos en una exposición lleva a una recomendación de un estudio más profundo o una publicación. Nada de eso sucede.

\noindent
Como resultados, se puede ofrecer uno muy rápido, que en apenas 15 minutos se puede solventar, que es respuesta a supuestos desafíos imposibles, que una vez resueltos, acaban en excusas para no admitir que minutos antes era algo que se creía imposible. En 15 minutos puedo empezar a ofrecer ``grietas'' en las demostraciones del teorema. Son demostraciones que están públicas y algunos divulgadores las consideran correctas, sin corrección hasta que se conoce el resultado propuesto, que hasta la fecha nadie ha podido negar.

\noindent
Casi todas las demostraciones tienen el mismo defecto: son técnicas reproducibles entre conjuntos que tienen el mismo cardinal. Lo cual las hace no confiables. Y eso se manifiesta de diversas formas, con fenómenos numéricos que ya han sorprendido a diversos matemáticos. La dificultad radica en que la capacidad para comprobar que ambos conjuntos tienen el mismo cardinal, varía. Por eso se ofrece una reproducción muy sencilla y rápida, de un caso muy simple, para demostrar que SUCEDE, y crear la duda sobre si sucede en más casos, ganando así el derecho a más tiempo y exposiciones más largas.

\noindent
La demostración definitiva, es ``observar'', como el cardinal de $\mathbb{P(N)}$ y $\mathbb{N}$ son indistinguibles, presentados en el formato adecuado. Y luego estudiar TODAS las demostraciones clásicas que afirman lo contrario. Y si hay curiosidad, aparte del experimento, en otro tiempo, hablar de resultados paralelos que reflejan una y otra vez lo mismo. Y la demostración consistirá en dos grupos de matemáticos diciendo que un conjunto es más grande que el otro y viceversa, con absoluta seguridad, sin ningún tipo de duda, hasta que oigan al otro grupo decir EXACTAMENTE lo contrario :D.

\noindent
La incredulidad, y la falta de fé en la propia razón y conocimientos, son reacciones muy comunes. Aquí dejo un texto que me envió por email un catedrático al que le realicé una exposición parcial de mi trabajo:\\\\
\\
``\\
Hace ya un par de años un compañero de la Universidad me pidió que atendiera a un amigo suyo, 
que tenía unos resultados matemáticos sobre los que quería saber mi opinión. Así conocí a un 
joven, Juan Carlos Caso Alonso, que sostenía que el teorema de Cantor es falso y que el conjunto 
potencia de cualquier conjunto infinito tiene el mismo cardinal que dicho conjunto. Le advertí que 
para mí, y para casi todos los matemáticos del Mundo, el resultado de Cantor es un teorema verificado.
Juan Carlos no tiene formación matemática, pero posee unas grandes inquietudes intelectuales y una 
gran capacidad de trabajo. En la primera reunión puso tal voluntad y tanto interés en el tema que 
decidí oírle más veces con el objetivo de encontrar el error en sus razonamientos. Y esa tarea, 
ciertamente, se me ha hecho muy difícil. No es fácil hallar el error. Algún resultado parcial, como la 
TIP, me resultó curioso e interesante, pero no puedo asegurar aún su validez. Tampoco sé cuál es 
el papel que juega en su demostración, porque no conozco la totalidad de la misma. Posiblemente 
cientos de páginas, con las dificultades que tiene de manejar el lenguaje matemático, frente a la 
sencilla prueba original de Cantor.\\
''\\
Palabras de José Manuel Méndez Pérez, catedrático de la ULL. Nos reunimos como unas 6 veces a lo largo de 3 años. Sus palabras son un ejemplo de como los resultados son difícilmente negables, y como la incredulidad, comprensible, impregna todas las reacciones. La TPI ``tan solo'' es una herramienta alternativa a las biyecciones, para comparar cardinalidades infinitas entre conjuntos. Y tengo una persona, con prestigio reconocido, diciendo claramente que le pareció una herramienta interesante. Y prometo que no fue un juez amable, fueron 6 reuniones en las que cuando encontraba fallos, los manifestaba sin piedad, hasta que los fui arreglando.

\noindent
Ahora debo añadir una anécdota, pero estas son MIS palabras, que con tiempo, puedo defender. La TPI completa se la he presentado a dos personas. Una de ellas, Pepe Mendez, como me deja llamarle, y otra una persona anónima de reddit. Ninguno de los dos, al acabar la definición de la TPI y su aplicación al caso $\mathbb{P(N)}$ vs $\mathbb{N}$, supo decir exactamente dónde estaba el fallo, pero ambos no estaban convencidos del todo, creían que el fallo DEBÍA estar en alguna parte. Lo gracioso, es que uno piensa que el error consiste en que estoy comparando, sin darme cuenta, $\aleph_{1}$ con $\aleph_{1}$. El otro piensa igual, solo que estoy comparando, sin darme cuenta, $\aleph_{0}$ con $\aleph_{0}$. Quitando que son tan inconfundibles que no saben bien cual escoger al ``suponer'' el fallo... NINGUNO de los dos dudó ni un segundo que estábamos hablando de conjuntos con el mismo cardinal.

\noindent
Puedo prometer tranquilamente que es un puzle tremendamente desafiante, porque tengo experiencia previa con expertos. Y se vuelve más interesante si hay más de uno presente, para escucharse entre ellos. Y no es lo único que he visto. La gente siempre resuelve esta anécdota, o parecidas, de la misma forma: `` ...a saber quienes serán esas personas...'' hasta que les toca a ellos tratar de resolver el puzle, y la actitud cambia de forma radical.

\noindent
Me he cruzado con gente que hasta ha llegado a citar mi propio trabajo, argumento por argumento... en lo que yo llamo, el ``cortocircuito''. Decirme que si fuese a la universidad, entendería mucho mejor ``mi propio trabajo''. Llegar a decirme a la cara que no tiene dudas que uno de mis antiguos trabajos es ABSOLUTAMENTE CORRECTO, al garantizarme la única pieza dudosa. El primer problema es que fue un trabajo descartado por ``crankery''. El segundo es que estaba de acuerdo en que si lo que decía no era cierto, el Teorema de Cantor era falso: por eso optó por esa opción, la otra desmontaba el teorema sin dudas.

\noindent
RESUMIENDO: Me he pasado 6 años chequeando, y revisando el trabajo. Cada punto es escogido por una persona diferente como incorrecto, de forma aleatoria, y a la vez, tengo varios matemáticos, que dicen que ese mismo punto es absolutamente correcto, pero escogen aleatoriamente otro punto, también aceptado por varias personas como obvio y trivial. O sea: sólido y correcto. Todos los puntos chequeados, pero de esa forma tan irregular. Por eso necesito varias personas.

\noindent
Para no tener estudios matemáticos, consigo desarrollar herramientas que les parecen interesantes a catedráticos. Y a más gente, pero son personas anónimas de internet. Uno decía tener 20 años de experiencia en teoría de conjuntos, y a una de mis herramientas la calificó de ``ingeniosa''.

\noindent
El trabajo ha alcanzado un estado de solidez tal, que ``confunde a matemáticos'' y les hace AFIRMAR lo mismo y lo contrario, en sus intentos desesperados de ``suponer'' el fallo. Porque claro, la observación final es IMPOSIBLE, así que el fallo debe estar en alguna parte. Pero ese camino les lleva a cuestiones muy extrañas, como afirmar lo que pretenden negar, sin darse cuenta.


\noindent
También se puede consultar con mi socio: Francisco Mario Cruz Almeida. Llamándole, en hora de Canarias y horario laboral, al 922 195005. Fue él quién me presentó a Pepe Méndez, después de un par de años presentándole resultados cada vez mejores. Tiene el trabajo un poco oxidado porque hace años que no lo mira, pero si os puede confirmar dos cosas. Una, que el, antes de conocerme, creía en la existencia de infinitos de diferentes tamaños. La segunda, que la semana pasada le ofrecí la ``biyección'' que lleva AÑOS pidiéndome... y me dijo: ``Es un buen punto de partida''.

\noindent
En 15 minutos puedo cambiarle la cara a un grupo de matemáticos sobre su fé en la inquebrantabilidad de la demostración. LA EXPERIENCIA me dice que hay dos grupos de matemáticos, cada uno usa una rama de la demostración y la creen perfecta y hermosa. Una, es la sencilla. Los que usan la segunda te lo dicen bien claro, que el fallo es que no has tenido en cuenta que la ``versión real'' es la que ellos dicen. Sin problemas... lo primero es pq al primer grupo NADIE les dice gran cosa ni se les puntualiza. O sea, es una demostración QUE CONVENCE A MATEMÁTICOS... y la de ellos tiene otros fallos. Se puede hacer una analogía rápida, para explicarles en que consiste, pero una vez acabado el experimento, como he tenido ya que desarrollar las herramientas necesarias, las puedo usar para indicarles una grieta GORDÍSIMA en esa técnica: lo que creían impredecible, es ABSOLUTAMENTE PREDECIBLE DE ANTEMANO. Pero claro, con mis alternativas a las biyecciones. Es un fenómeno numérico muy curioso... pero solo demuestra o apunta, a que las demostraciones son incorrectas, no a que el teorema sea falso. Por eso el experimento es más contundente.



