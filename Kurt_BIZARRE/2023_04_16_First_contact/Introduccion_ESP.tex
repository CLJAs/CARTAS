\chapter{Introducción}
	\noindent
	Lo que me pides... ¿Cuáles serían las palabras: 'No es justo´? No solo me ha costado 25 años desarrollar el contraejemplo, con cada punto chequeado por otra gente y con referencias de dos matemáticos. Encima me pides que te diga el fallo de la demostración, cuando el teorema 1, ($|A| < |P(A)|$), no admite excepciones del tipo $ |\mathbb{N}| = |P(\mathbb{N})| $.\\\\
	
	\noindent
	Es una trampa circular por varios motivos:\\\\
	1) Tengo que resumir y descartar material, MUCHO, porque tengo una ventana de atención por tu parte muy reducida respecto a lo que necesito en realidad. No puedo hacerlo en el formato que me resulta más... 'eficiente´. Y no puedo cometer ni el más mínimo fallo. Y es un campo de minas, y te pido perdón de antemano, porque también debo evitar que 'cortocircuites´. Pensarás que soy un flipado, que soy un ignorante y que digo eso porque no entiendo las críticas que me hacen, pero no es el caso. Como dudo de todo, hasta de mi mismo, suelo usar el feedback que me dan, y descubro 'reacciones curiosas´ cuando unos matemáticos me dicen lo que piensan sinceramente, sobre lo que me dicen OTROS matemáticos... pensando que son ideas mías. Tengo muchísimas anécdotas, de diferente índole.\\\\
	2) Necesito el contraejemplo para que entiendas el fallo de la lógica. Como ya te dije, tengo tiempo reducido y NO puedo explicarte TODO lo que tengo, y no me vas a creer si solo te menciono la existencia de lo que es una 'paradoja híbrida´... :D... Me la voy a jugar, pero... ¿A que te estás preguntado que coño es una paradoja híbrida? :D... Tienes que verlas 'actuar´ para creer que existen. PERO no me dejas usar el contraejemplo completo (4 horas y media delante de una pizarra), y como muchos antes, me pides que te explique el fallo de la demostración, que es, PRECISAMENTE, el contraejemplo (entre otros, pero que tampoco sirve de nada decírtelos si no ves como falla el teorema con tus propios ojos).\\\\
	3) El contraejemplo SÉ que lo tengo más que doblemente chequeado. La explicación del fallo no. No paro de pedir ayuda y no la consigo, a pesar de los resultados que ofrezco. VOY A INTENTAR crearte una 'analogía´ más sencilla y rápida del caso real, PERO RECUERDA, el fallo en la lógica de la demostración, y el contraejemplo, están relacionados, \textbf{\underline{PERO NO SON LO MISMO}}. Puedo construirte relaciones, pero igual meto la pata en terminología de lógica.\\\\
	
	\noindent
	Te pido que no olvides que me dejo mogollón de cosas en el tintero. La TPI tiene mogollón de propiedades, pero no te las voy a decir porque el documento donde me explayé tiene 60 páginas. A Don Pepe Mendez, tardé hora y media delante de una pizarra en explicársela:\\
	https://vixra.org/abs/2209.0120\\
	No me hagas lo que me hizo una persona anteriormente. Primero alucinó con que se pudiesen construir. Luego vino a los dos días y me dijo: mira, puedo construir una entre $\mathbb{Q}$ y $\mathbb{R}$, y como sabemos que ambos tienen cardinal diferente, es irrelevante que existan las TPIs... Lo primero es que YO acepto que se puedan aplicar a $\mathbb{Q}$ y $\mathbb{R}$, ya que tienen el mismo cardinal y es una técnica que nace con intención de generalidad. Básicamente me dice que puede replicar mis resultados y que eso está mal :D. SEGUNDO, no sé si lo has visto pero eso es un argumento circular. Usa el Teorema, para negar un contraejemplo del teorema, para evitar juzgar las repercusiones y las propiedades de las TPIs... por eso te digo que la gente 'cortocircuita´. Si pudiésemos usar un teorema, para negar cualquier contraejemplo, sin más argumentos, se podría demostrar cualquier cosa, pues siempre te puedo decir que lo tuyo es irrelevante ya que el Teorema es cierto.\\\\
	
	\noindent
	Te puedo decir que soy el hombre más guapo del mundo. Th :D. Y tú me puedes enseñar una foto de Brad Pitt. Así que te respondo: ¿Sabes qué, como ya sabemos que soy el hombre más guapo del mundo, ni siquiera le vamos a enseñar esa foto a mi novia, y la vamos a quemar, okey?\\\\
	
	\noindent
	Vamos con la analogía rápida de un caso más sencillo que el real.
	
	