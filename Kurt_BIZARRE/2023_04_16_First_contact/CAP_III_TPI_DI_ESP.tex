\chapter{DI, TPI, y el fenómeno actual que me está sucediendo}

	Ambas, la DI y la TPI, son 'juegos cardinales´. La DI es más vieja y es menos 'rigurosa´ en su exposición. Las dos parten del supuesto, que si el conjunto que hace el papel de Dominio, es ABSURDAMENTE mucho mayor que el otro, si su diferencia cardinal es INCONMENSURABLE, :D, no deberían poder construirse con éxito. Y si se construyen con éxito, ambas implican que el cardinal del Dominio, NO es mayor que el cardinal del otro conjunto (que no tiene porque ser el conjunto Imagen, ojo). Lo cual implica que pueden ser iguales.

	\section{DI}
		
	
	
	
	\section{TPI}
	
	
	
	\section{Resultado increíble}