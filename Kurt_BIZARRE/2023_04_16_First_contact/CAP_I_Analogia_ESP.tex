\chapter{Analogía rápida del fallo lógico de las demostraciones}

	\section{La Tabla}
	\begin{table}[h!]
		\begin{tabular}{|c|c|c|c|c|c|}
			\hline
			$O\_o!!$ & $R_{1}$ & $R_{2}$ & $R_{3}$ & $R_{4}$ & $R_{5}$ \\
			\hline
			$A$ & $B_{1}$ & $B_{2}$ & $B_{3}$ & $B_{4}$ & $B_{5}$ \\
			\hline
			\hline
			$1$ & $\{1\}$ & $\{1\}$ & $\{1\}$ & $\{1\}$ &$\{1\}$  \\
			\hline
			$2$ & $\{2\}$ & $\{2\}$ & $\{2\}$ & $\{2\}$ &$\{2\}$\\
			\hline
			$3$ & $\{3\}$ & $\{3\}$ & $\{3\}$ & $\{3\}$ &$\{3\}$\\
			\hline
			$4$ & $\emptyset$ & $\{4\}$ & $\{5,6\}$ & $\{4,5,6\}$ & $b=\{a \in A | a \notin R_{5}(a) \}$ \\
			\hline
			$5$ & $\{6\}$ & $\{6\}$ & $\{6\}$ & $\{6\}$ &$\{6\}$\\
			\hline
			$6$ & $\{5\}$ & $\{5\}$ & $\{5\}$ & $\{5\}$ &$\{5\}$ \\
			\hline 
		\end{tabular}
	\end{table}

	\section{Explicación de la tabla}
	
	\noindent
	La tabla representa un 'asedio´ a una paradoja híbrida, muy simple. Son cinco relaciones\footnote{
		RARAMENTE voy a usar la palabra 'función´, primero porque es correcto llamarlas también relaciones, antes de ser función debes ser relación. Segundo porque mi trabajo se basa en relaciones que NO son aplicación (recuerda, todo está chequeado, lo hago de forma correcta hasta los puntos débiles que te mencioné en el documento de tres páginas): 
		$\href{https://docs.google.com/document/d/1DwgidmjAFIlmpDw0rucYLSeTMFiOuFxfVQxPFC-6vwI/edit?usp=share_link}{Documento}$
	}. De $R_{1}$ a $R_{5}$ son relaciones entre A, que tiene cardinal 6, y conjuntos de subconjuntos de A, los $B_{i}$, con cardinal 6, también. Cada columna, después de la doble línea, representa los elementos de cada conjunto. La tabla indica que el elemento del conjunto $B_{i}$, está relacionado mediante $R_{i}$, con el elemento de A que está en la misma fila.\\\\

	\noindent
	No sé por qué la gente reacciona airadamente al ver este ejemplo, pero necesito que sea simple para que se vea muy claro varias cosas. Sé de sobras que los conjuntos implicados sólo tienen cardinal 6, y que no son $\mathbb{N}$ ni $P(\mathbb{N})$. Te dije que sería una analogía con un caso más simple.\\\\
	
	\noindent
	Podríamos demostrar que $B_{5}$ NO es sobreyectiva, si 'confiamos´ en la técnica de Cantor. Al crear el par $(4, b)$ en $R_{5}$ creamos la doble contradicción que es núcleo de su demostración (junto con la definición de 'b´). Y a la misma vez, sabemos, gracias a que es muy simple el ejemplo, que A y $B_{5}$ tienen el mismo cardinal. Finito, pero el mismo cardinal.
	
	\noindent
	La demostración habla de 'cualquier relación/función posible entre A y P(A)´, y aquí tenemos un caso EXTREMADAMENTE particular, donde la conclusión de la demostración, cojea. Okey, no son esos conjuntos, pero tienen el mismo cardinal, permiten definir 'b´ y generar la doble contradicción. \\\\
	
	\noindent
	No solo eso. Y esta es la parte del 'asedio´. $B_{5}$ es uno de los otros $B_{i}$ por narices: o es $B_{1}$, o $B_{2}$, o $B_{3}$, o $B_{4}$. Ya que la definición de 'b´ DEBERÍA coincidir con 'alguno´ de los elementos que están en su misma fila, el resto de elementos son idénticos. Eso \textbf{\underline{SI}} 'b´ EXISTIESE y fuese un subconjunto de A. Esto implica que $R_{5}$, en realidad, es una de las otras $R_{i}$... que no generan dobles contradicciones y sí son biyectivas.\\\\
	
	\noindent
	Fíjate lo curioso que es: al cambiar la definición del subconjunto hemos desactivado la doble contradicción, pero usando EL MISMO subconjunto. No pasa nada por crear los pares:\\
	$(4, \emptyset)$, \\ 
	$(4, \{4\})$,     \\
	$(4, \{5,6\})$ y  \\
	$(4, \{4,5,6\})$  \\
	Lo malo es que no sabemos exactamente CUAL de ellos es 'b´, y a la misma vez, debe ser alguno de ellos. YO, no lo tengo claro, al menos :D. Pero gracias a que el ejemplo es tremendamente sencillo, sabemos con certeza que de $R_{1}$ a $R_{4}$ son todas relaciones, funciones, biyectivas.\\\\
	
	\noindent
	¿Lo experimentas? Este es el sabor de boca que te deja un asedio para atacar una paradoja híbrida. Claro, en su versión light. Yo estuve sin dormir tres días... pero igual tú te libras. Otra gente se cabrea. O cortocircuita.\\\\
	
	\subsection{Punto 1}
	
	\noindent
	No sé como expresar esto de forma 'formal´: las diagonalizaciones son IRRELEVANTES, pq se pueden construir sobre conjuntos con el mismo cardinal. Pero esa afirmación no me la he sacado de la manga, como me hicieron con la TPI, que SI busca poder crearse entre conjuntos con... 'el mismo cardinal infinito´. Con una simple observación de la tabla puedes coincidir conmigo: los elementos de cada conjunto se pueden contar a ojo, si hubiese dudas :D.\\\\
	
	\subsection{Punto 2}
	
	\noindent
	2) Me dirás que me queda la de $\mathbb{N}$ vs $\mathbb{R}$ ( o cadenas binarias infinitas representando el conjunto de Cantor), que no me la has pedido. El Teorema 2 de tu lista:\\
	$|P(\mathbb{N})| = |\mathbb{R}|$\\
	No contradice mis afirmaciones. Y este que te sugiero lo tengo analizado en estas dos horas de vídeos de youtube:\\
	\href{https://www.youtube.com/watch?v=reRUUKGFXf0&list=PLcEv5UNDUdw68yFXf2kYGDZVyIGpCfGdy}{Lista de videos}\\
	TE ADVIERTO, mi material es caótico. Solo te lo pongo para que veas como 'tengo algo para eso´. He aprendido a ordenarlo un poco todo y lo estoy reescribiendo. Esos vídeos necesitan unos pequeños apaños. Es una rama no chequeada todavía. Se repiten montón de cosas porque hay que explicar las definiciones y propiedades comunes en cada rama, si empiezo de cero con alguien... Te vas a quedar a cuadros, pero básicamente puedo PREDECIR cualquier elemento externo, creado con cualquier técnica de diagonalización, CUALQUIERA, para cualquier intento de biyección POSIBLE entre $\mathbb{N}$ y $P(\mathbb{N})$, con relaciones PREVIAMENTE establecidas. ¿Qué apaños?¿Qué significa 'predecir´?¿Cómo que uso más de una relación?... :D. Por eso sólo quiero decirte que ''existe´´.
	
	\noindent
	\textit{NOTA: perdona que me ponga quizás prepotente. Sé que hay más demostraciones... YO NO SOY matemático. UN SOLO contraejemplo las invalida TODAS. Mencionarlo es indigno de un matemático de carrera. Es mucho morro pedirme que arregle TODAS. YO SOLO. Ni siquiera las conozco. He visto alguna y 'creo´ saber por dónde van los tiros del error... pero una vez más, no paro de pedir ayuda, y como no soy nadie... pues eso.}\\\\
	
	\noindent
	Salta aquí al punto 3, el resto de esta subsección solo lo escribo para dejar constancia. Necesitas mogollón de contexto para entender los apaños, aunque, por si los lees, trataré de explicarte 'algo´.
	
	\subsubsection{Añadido ignorable solo para dejar constancia}
	
	\noindent
	Este punto me está retrasando así que voy a ser muy informal con él (recuerda que es ignorable):\\
	1) Simplemente que en los vídeos cuando dice ``universo'', a veces quiere decir $r_{\theta_{k}}$. Sobre todo en las tablas de la flja\_abstracta y cuando dice ``resuelve''.\\\\
	2) Para comprender uno de los puntos de desescalado de $\aleph_{1}$ a $\aleph_{0}$ hay que mirar en el documento de la TPI QUÉ son las Familias de parejas de D-pares (que se crean clasificándolos por su valor $\gamma$).\\\\
	3) Pensar que las relaciones en paralelo PREDEFINIDAS son lo mismo que intentar una biyección, montar un elemento externo y luego crear otra relación que añada ese nuevo elemento externo.. y así sucesivamente, es de una vagancia intelectual brutal. No sólo están PREDEFINIDAS, hace años... sino que al tener cada una un pedacito de $LCF_{2p}$ diferente y disjuntos con los demás pedacitos... es el buscador QUIEN DEBE ahora encontrar un elemento externo a TODAS las relaciones, no solo a algunas relaciones. Y AHORA SI, se entiende el juego del vídeo, pues NO EXISTE un elemento externo que no esté cubierto por el sistema de relaciones $r_{\theta_{k}}$. Sería una vergüenza, QUE SUCEDE, no poder encontrar un elemento externo que no pueda escapar de un sistema de subconjuntos con cardinal $\aleph_{0}$ X $\aleph_{0}$. Y al ser paralelas, existir todas al mismo tiempo, y tener conjuntos Imagen disjuntos, podemos disponer de ellas como queramos ``en paralelo''. De ahí lo de poder asignar varios Packs. También los podemos seleccionar porque el elemento externo debe serlo A TODO el sistema... ya que es una mera partición de un conjunto con cardinal $\aleph_{0}$. Como debe ser EXTERNO A TODO, puedo escoger a capricho las relaciones INAMOVIBLES y PREDEFINIDAS que más me gusten. Es como... una red que atrapa al elemento externo y no le deja escapar... solo que en ``multiversos''. NO ESCAPAR en uno solo... lo convierte en un caso predecido previamente, incluso antes de que se te ocurriese empezar a crearlo. Uses la técnica que uses. En el vídeo me permito predecir dos técnicas diferentes a la vez, porque en realidad me da igual como se crean... si un $SNEI$ existe, tiene un valor $\gamma$ con ABSOLUTAMENTE todos los demás SNEIs. YA ESO, desactiva todas las demostraciones de Cantor y sus técnicas y me permite predecirlo, como demuestro en el juego de los vídeos Y NADIE me ha dicho que sea incorrecto. Incluso hay gente que le ha sorprendido el resultado... pero... acaban confundiendo el caso de ir cambiando la definición completa de la relación para adaptarse al elemento externo... con la opción de tener un sistema PREDEFINIDO de relaciones que no alteran el cardinal del conjunto y que predicen TODOS los elementos externos. Cuando digo TODOS es TODOS. Solo por existir, ya se puede predecir... gracias a su propiedad $\gamma$. Al nacer, está condenado al fracaso :D. Si NO existe, no debo preocuparme por él. Si NO EXISTE, no es un SNEI jajajaja, me da igual que no tenga propiedad $\gamma$.
	
	\noindent
	Lo malo de los escritos es que me llevan mucho tiempo. Perdona. Me quedan largos porque me cuesta explicarme de forma concisa y no puedo verte la cara para ver si es suficiente con lo que he dicho o no. Por ejemplo, lo siguiente, lo diría o no dependiendo de tu cara. Esto lo estoy escribiendo al final, recuerdo haberlo escrito, así que no te extrañe que lo haya escrito dos veces y esté repetido en alguna parte del documento.
	\\\\
	
	\noindent
	Imagina las batallas cardinales, como batallas que se ganan por número de soldados. Y las estrategias son relaciones (funciones o no). Un soldado cuando mata a otro, el también desaparece. El Naive CA theorem busca relaciones 1:X. El elemento del dominio tiene varios candidatos que le pueden hacer desaparecer, pero los soldados del conjunto Imagen solo tienen un candidato cada uno, desde su punto de vista.
	\\\\
	
	\noindent
	Ahora imagina que defiendes una ciudad. El ejército enemigo debe cruzar unas montañas, que tienen diferentes pasos de montaña posibles. Como no quieres que lleguen a los muros de tu ciudad, divides tu ejército en divisiones, y mandas a cada una a defender cada paso. Deseas librar la batalla lejos de tu ciudad. ``Pueden haber'' varias batallas diferentes, cada una con su estrategia; cada una con su relación; cada una con su $r_{\theta_{k}}$... Todo depende del paso de montaña que escoja el enemigo. Da igual si la que sucede de verdad, la pierdes o la ganas: lo importante aquí es que hacer divisiones de tu ejército y plantear una estrategia diferente para cada división, no es hacer trampas con el cardinal de tu ejército. ¿Estamos de acuerdo? Espero que sí. Por ESO, el sistema de relaciones es diferente a ir cambiando la relación según te van creando un elemento externo nuevo cada vez.
	\\\\
	
	\noindent
	Ahora imagina, que en vez de tener el escenario de varios pasos de montaña, el enemigo está enfrente de las puertas de tu ciudad, y sacas a tu ejército a pelear. No quieres que la batalla sea dentro. AHORA, puedes crear divisiones, o mejor dicho en este escenario, ``lineas de defensa''. YA hemos dicho que es legal crear divisiones, lo único que hacemos es usar TODAS las divisiones a la vez. SI EL ENEMIGO, quiere decir que te ha derrotado, DEBE derrotar TODAS tus divisiones. Se supone que tus divisiones son una partición de tu ejército original. Si quiere poder decir que ha  derrotado a TODO tu ejército, debe poder derrotar a TODAS tus lineas de defensa.. A TODAS, sin dejarse ni una.
	\\\\
	
	\noindent
	Y aquí tenemos OTRA FORMA en la que falla la diagonalización. Aparte de la tabla inicial de este documento que deja al descubierto la paradoja híbrida. Si vieses esas dos horas de vídeos... aparte de no necesitar que te explique el contexto de los dos juegos, pq ya conocerías el contexto que tienen en común...descubrirás una forma de PREDECIR todos los elementos externos. TODOS, para TODO posible intento de biyección. Pero no tenemos UNA biyección, tenemos un sistema de relaciones no aplicación PREVIAMENTE DEFINIDAS (las $r_{\theta_{k} }$ ). Y da igual COMO intentes crear CUALQUIER elemento externo... ninguno sobrevive a todas las filas de defensa... :D. En el último vídeo se explica que Packs son los que derrotan a cualquier elemento externo, para cualquier intento de biyección. PUEDO ELEGIR, pq me he molestado PREVIAMENTE en tener MÁS DE UNA OPCIÓN de relación. 
	\\\\
	
	\noindent
	El juego de intentar definir una biyección, crear un elemento externo, volver a crear otra que lo añada, crear otro elemento externo, crear OTRA biyección... en nuestro escenario sería el equivalente a tener UNA SOLA linea de defensa: una sola relación. Y cada vez que nos derrotan, un nigromante resucita a todo nuestro ejército, solo para volver a ser derrotado. Una y otra vez. Y encima es adaptativo, no predictivo... vamos definiendo cada nueva relación, según nos van proponiendo nuevos elementos externos no cubiertos. Mi técnica usa relaciones, tan previamente definidas... que no les he cambiado una sola coma desde hace años. :D... ni la cambiaré, como explico en el vídeo, para cualquier elemento externo que se te ocurra mencionar. Si la biyección que intentaste tenía el formato:\\
	$ f: \mathbb{N} \longrightarrow P(\mathbb{N})$   ( O $LCF_{2p}$ vs SNEIs) \\
	Y llamamos a cualquier elemento que se te ocurra crear X.\\
	Para la unión del conjunto Imagen de esa biyección, UNIÓN, X, mi sistema de relaciones ofrecen Packs disjuntos de números naturales de cardinal infinito... PARA TODOS LOS MIEMBROS DE ESA UNIÓN, incluido X... incluso antes de saber que X ibas a crear. INCLUSO, si alguien decide crear OTRA biyección que añada X, y crear a su vez, un nuevo elemento externo, mi sistema TAMBIÉN lo predice.
	\\\\
	
	\noindent
	RESUMEN: los elementos externos son PREDECIBLES solo con $\aleph_{0}$ elementos. Solo hay que saber jugar nuestras cartas MEJOR que como se les ha ocurrido a otra gente antes. RECUERDA... usar relaciones no aplicación va en contra de la definición de cardinalidad... pero ESPERO que estuvieses de acuerdo con las propiedades de las batallas cardinales y sobre la legalidad de crear divisiones y tener más de una opción de batalla. PARA COMPRENDER POR QUÉ se pueden asignar Packs disjuntos, sin conocer X previamente, necesitas ver las dos horas de vídeo. De todas formas... entra dentro de las 4 horas de exposición de los dos juegos. No explicar la predicción... sino las condiciones que lo permiten. Pq permiten también crear los dos juegos.
	\\\\
	
	\subsection{Punto 3}
	
	\noindent
	Ya sé, ya sé... el ejemplo es demasiado sencillo. PERO, si ya se dá en UN caso ¿Sucederá en más? Y la gracia es que sucede para $\mathbb{N}$ vs $P(\mathbb{N})$.\\\\ 
	
	\noindent
	En este ejemplo se vé fácil porque los conjuntos son muy sencillos. El asedio sólo consiste en crear todas las alternativas posibles a la función característica de 'b´, con descripciones que solo listan directamente los elementos. Cambiamos su función característica, SIN CAMBIAR el subconjunto final: 'alguna de ellas´ es una función característica\footnote{Espero no equivocarme. La función que dice si un elemento pertenece o no al conjunto} equivalente a la de 'b´. Las dos generan el mismo subconjunto de A. Si existiese de verdad.\\\\
	
	\noindent
	En el caso real, no solo vamos a cambiar TODAS las funciones características de TODOS los elementos de $P(\mathbb{N})$. Vamos a crear alternativas al concepto de ``función'' (relaciones no aplicación que van a ser ``útiles''); al concepto de inyectividad o biyectividad para comparar cardinalidades infinitas; vamos a poner en duda la definición de ``cardinalidad''... e incluso puede que algún axioma ZF... vamos a ``intuir'' que existen fenómenos lógicos nuevos (las Paradojas Híbridas\footnote{Una P.H. es un absurdo metido con calzador, que es un absurdo en sí mismo, al ser paradoja, que en realidad no está relacionado con la frase original de la demostración. Para pequeña muestra, un pequeño botón... $R_{5}$ debería ser biyectiva, a pesar de la doble contradicción que podemos generar gracias a la definición de 'b´. Se llaman híbridas, pq tienen muchos estados diferentes, según la relación de la que dependen, es un buen subconjunto, o se convierte en paradoja... y nos permite decir que hemos llegado a un absurdo, cuando lo único que hemos hecho es añadir una paradoja por la cara en mitad de la demostración.}). Y lo que no me esperaba, vamos a expandir el asedio a dos técnicas diferentes. Cuando presento la TPI, para negarla, me GARANTIZAN justo el punto que me faltaba de la DI, y para negar la DI, me GARANTIZAN justo el punto que me faltaba en la TPI. ¿Cuál de las dos es la correcta? ASEDIO!! Jajajaja. Las dos dependen de interpretaciones complementarias del mismo tipo de intersecciones infinitas... para cada posible interpretación, tengo una construcción de ``equivalente'' a una relación inyectiva imposible según Cantor. Aunque no sepa decirte cuál es la interpretación correcta de las dos.\\\\

	\noindent
	¿Se pueden construir diagonalizaciones? Sí, pero si no me das 4 horas y media, me niegas el contraejemplo, me niegas el poder enseñarte el equivalente a poder ver que todos los conjuntos implicados SIEMPRE han tenido cardinal 6 (en el ejemplo de la tabla).\\\\
	
	\noindent
	No nos podemos saltar el orden. Para que entiendas el fallo de las demostraciones, necesito enseñarte el contrajemplo primero. Esa es mi serpiente que se muerde la cola.	Todo el mundo me insiste en que si el contraejemplo existe, les indique el fallo de las demostraciones. Y eso es imposible, al menos para mí. NECESITO el contraejemplo. Debes entender que las diagonalizaciones son irrelevantes y que petan de muchas maneras diferentes si las estudias a fondo.
	\\\\
	
	\noindent
	\textit{Aquí ya me meto droga dura en vena: no me hagas mucho caso :D. Una paradoja híbrida no es una función característica válida de un subconjunto, no es un Teorema, y no es un programa (Cantor, Gödel, Turing...). Por eso hay que cambiar un axioma de ZF, para añadir de alguna manera la coletilla de que no puede crearse un subconjunto con una paradoja híbrida. Hay que estudiarlas a fondo, para mejorar su detección y definición, por si también afectan a los teoremas de Gödel, Turing... y alguno de Teoría de Lenguajes Formales. O por si hay más casos...}\\\\
	
	

	
	
	
	
	
	
	
	



